\documentclass[]{article}

% Package `amsthm` and `thmtools` must come before package `hyperref`.
\usepackage{amsthm}
\usepackage{thmtools}
% Package `hyperref` must come before package `complexity`.
\usepackage[pdftitle={Restricted probabilistically checkable proofs}, pdfauthor={Jeffrey Finkelstein}]{hyperref}
\usepackage{complexity}
\usepackage{amsmath}

\newcommand{\gen}[1]{{\langle #1 \rangle}}

\declaretheorem[]{theorem}
%\declaretheorem[numberlike=theorem]{corollary}
%\declaretheorem[numberlike=theorem]{lemma}
\declaretheorem[numberlike=theorem]{proposition}
\declaretheorem[numberlike=theorem, style=definition]{definition}

\title{Complexity of computing minimum generating sets for groups}
\author{Jef{}frey~Finkelstein}
\date{\today}

\begin{document}
\maketitle

In this work we show that the problem of computing the minimum generating set of a group is decidable in deterministic polynomial time.

%% A \emph{group} is a two-tuple $(G, \cdot)$ where $G$ is a (finite or infinite) set and $\cdot$ is a binary operation on $G$ satisfying closure, associativity, and invertibility.
%% The \emph{Cayley table} of a group is a matrix whose rows and columns are labelled by elements of $G$, and where entry $x$ is labelled by $(a, b)$ if $a \cdot b = x$.

We denote by \P{} the class of decision problems decidable by a deterministic Turing machine halting in polynomial time.
We denote by $\L^2$ the class of decision problems decidable by a deterministic Turing machine using at most $O(\log^2 n)$ space.
The (inclusion) relationship between \P{} and $\L^2$ is unknown.

If $(G, \cdot)$ is a group and $S$ is a subset of $G$ then the set of words on $S$ is denoted $S^*$ and the set of inverses of elements in $S$ is denoted $S^{-1}$.
If $T \subseteq G$, the set \emph{generated} by $T$, denoted $\gen{T}$, is defined by
\begin{equation*}
  \gen{T} = \{ x \in G \, | \, x \text{ is the product of elements in } T \}.
\end{equation*}
A \emph{generating set} for $(G, \cdot)$ is a set $T \subseteq G$ such that $\gen{T} = G$.

Some applications may require computing the minimum generating set for a given group.
We define a formal optimization problem in order to study its complexity, as well as the corresponding ``budget'' problem (a characterization of the optimization problem as a decision problem).

\begin{definition}[\textsc{Minimum Group Generating Set}]
  \mbox{}

  \textbf{Instance:} finite group $(G, \cdot)$ (given as a Cayley table).

  \textbf{Solution:} finite set $T \subseteq G$ such that $\gen{T} = G$.

  \textbf{Measure:} $|T|$.

  \textbf{Type:} $\min$.
\end{definition}

\begin{definition}[\textsc{Minimum Group Generating Set Budget}]
  \mbox{}

  \textbf{Instance:} finite group $(G, \cdot)$ (given as a Cayley table), natural number $k$.

  \textbf{Question:} Does there exist a $T \subseteq G$ such that $\gen{T} = G$ and $|T| \leq k$?
\end{definition}

In order to show that the \textsc{Minimum Group Generating Set Budget} problem is decidable in deterministic polynomial time, we reduce it to the \textsc{Minimum Group Rank Budget} problem, which is known to be \P-complete.

\begin{definition}[\textsc{Minimum Group Rank}]
  \mbox{}

  \textbf{Instance:} finite set $S$, finite set $U \subseteq (S \cup S^{-1})^*$.

  \textbf{Solution:} finite set $T \subseteq \gen{U}$ such that $\gen{T} = \gen{U}$.

  \textbf{Measure:} $|T|$.

  \textbf{Type:} $\min$.
\end{definition}

\begin{definition}[\textsc{Minimum Group Rank Budget}]
  \mbox{}

  \textbf{Instance:} finite set $S$, finite set $U \subseteq (S \cup S^{-1})^*$, natural number $k$.

  \textbf{Question:} Does there exist a $T \subseteq \gen{U}$ such that $\gen{T} = \gen{U}$ and $|T| \leq k$?
\end{definition}

Previously, the \textsc{Minimum Group Generating Set Budget} problem was known to be in $\L^2$, the class of problems decidable by a deterministic Turing machine using at most $O(\log^2 n)$ space \cite{lsz77} (see \cite[Proposition~3]{at06} for a brief description of the algorithm; I can't find a copy of \cite{lsz77} online).
%% A related problem, the \textsc{Quasigroup Isomorphism Problem}, is known to be in $\GC(O(\log^2 n), \FOLL)$ where $\GC(O(f(n)), \mathcal{C})$ is the class of languages decidable by a $\mathcal{C}$ algorithm augmented with $O(f(n))$ nondeterministic bits and $\FOLL$ is the class of languages decidable by a uniform family of circuits of polynomial size, unbounded fan-in, and depth $O(\log \log n)$.

\begin{theorem}
  $\textsc{Minimum Group Generating Set Budget} \in \P \cap \L^2$.
\end{theorem}
\begin{proof}
  We show a logarithmic space many-one reduction from \textsc{Minimum Group Generating Set Budget} to \textsc{Minimum Group Rank Budget}.
  Since \textsc{Minimum Group Rank Budget} is in \P{} (in fact, it is complete for \P{} under logarithmic space many-one reductions \cite[Problem~A.8.11]{ghr95}) and \P{} is closed under polynomial time many-one reductions, we conclude that \textsc{Minimum Group Generating Set Budget} is also in \P.
  The reduction is the mapping $\langle (G, \cdot), k \rangle \mapsto \langle G, G, k \rangle$.
  First, we observe that $\langle G, G, k \rangle$ is a correctly formatted instance of the reduced problem: $G$ is a finite set and $G \subseteq (G \cup G^{-1})^*$.

  Suppose that $\langle (G, \cdot), k \rangle \in \textsc{Minimum Group Generating Set Budget}$, so there exists a $T \subseteq G$ such that $\gen{T} = G$ and $|T| \leq k$.
  Since $G$ is closed under $\cdot$, we know $\gen{G} = G$ and hence $\gen{T} = \gen{G}$.
  Therefore we conclude that $\langle G, G, k \rangle \in \textsc{Minimum Group Rank Budget}$.
  For the converse, suppose that $\langle G, G, k \rangle \in \textsc{Minimum Group Rank Budget}$, so there exists a $T \subseteq \gen{G}$ such that $\gen{T} = \gen{G}$.
  As stated above, $\gen{G} = G$, so $\gen{T} = G$.
  Therefore $\langle (G, \cdot), k \rangle \in \textsc{Minimum Group Generating Set Budget}$.

  We have shown a correct logarithmic space many-one reduction to a problem in \P, and we conclude that $\textsc{Minimum Group Generating Set Budget} \in \P$.
\end{proof}

In this section we provide a clearer presentation of the proof of \cite[Theorem~7]{at06}.

\begin{proposition}
  $\textsc{MIN-NILP-GROUP-GEN} \leq_{ctt}^P \textsc{MIN-p-GROUP-GEN}$.
\end{proposition}
\begin{proof}
  Define the reduction by
  \begin{equation*}
    \langle G, k \rangle \mapsto \bigwedge_{i=1}^m \langle S_{p_i}, k \rangle
  \end{equation*}
  for all nilpotent groups $G$ of order $n$ and all postive integers $k$, where the prime factorization of $n$ is $p_1^{e_1}p_2^{e_2}\dotsb p_m^{e_m}$ and $S_{p_i}$ is the unique $p_i$-Sylow subgroup of $G$.
  Since the length of the input is $n^2$, the prime factorization of $n$ can be computed in polynomial time with respect to the length of the input.
  For each $i$, the Cayley table of the $p_i$-Sylow subgroup of $G$ can be computed in polynomial time because \ldots.
  The number of clauses in the conjunction is $m$, which is the number of prime factors of $n$, which is bounded by a polynomial in $n^2$ because \ldots.
  Hence the reduction is polynomial time computable.

  Suppose $\langle G, k \rangle \in \textsc{MIN-NILP-GROUP-GEN}$.
  \ldots
\end{proof}

\begin{proposition}
  $\textsc{MIN-p-GROUP-GEN} \leq_m^P \textsc{MIN-ELEM-ABEL-GROUP-GEN}$.
\end{proposition}
\begin{proof}
  Define the reduction by
  \begin{equation*}
    \langle G, k \rangle \mapsto \left\langle \frac{G}{\Phi(G)}, k \right\rangle
  \end{equation*}
  for all $p$-groups $G$ of order $n$ and all positive integers $k$, where $\Phi(G)$ is the Frattini subgroup of $G$.
  By \autoref{lem:frattinip}, the Cayley table of $\Phi(G)$ is computable in polynomial time.
  By \autoref{lem:factorgroup}, the Cayley table of $\frac{G}{\Phi(G)}$ is computable in polynomial time.
  Hence the reduction is polynomial time computable.

  \ldots
\end{proof}

\bibliographystyle{plain}
\bibliography{references}

\end{document}
