\documentclass{article}

% Package `amsthm` and `thmtools` must come before package `hyperref`.
\usepackage{amsthm}
\usepackage{thmtools}
% Package `hyperref` must come before package `complexity`.
\usepackage[pdftitle={Restricted probabilistically checkable proofs}, pdfauthor={Jeffrey Finkelstein}]{hyperref}
\usepackage{complexity}
\usepackage{amsmath}
\usepackage{amssymb}

\newcommand{\gen}[1]{{\langle #1 \rangle}}
\newcommand{\email}[1]{\href{mailto:#1}{\nolinkurl{#1}}}

\declaretheorem[numberwithin=section]{theorem}
\declaretheorem[numberlike=theorem]{conjecture}
\declaretheorem[numberlike=theorem]{lemma}
\declaretheorem[numberlike=theorem]{proposition}
\declaretheorem[numberlike=theorem, style=definition]{definition}

\title{Complexity of computing minimum generating sets for groups}
\author{Jef{}frey~Finkelstein}
\date{\today}

\begin{document}
\maketitle

\section{Introduction}

In this work we study the problem of computing the size of the minimum generating set of a group when given as a Cayley table, as well as the restriction of this problem to restricted classes of groups.
Previously, the best upper bound for this problem was $\L^2$ (also known as $\DSPACE(\log^2 n)$), due to \cite{lsz77} (also described in \cite[Proposition~3]{at06}).
The more recent work by many authors on ``bounded nondeterminism'' (in which an algorithm is allowed to nondeterministically guess a limited amount of bits and then verify the guess deterministically) as well as improved algorithms for deciding group membership due to \cite{bm89} and \cite{bklm01} allow us to improve this upper bound to $\GC(\log^2 n, \L)$ (defined in \autoref{sec:prelim}) using a more careful analysis of the original $\L^2$ algorithm.

The related problem of deciding group membership was placed in $\L^2$ by \cite{lsz77}, and later improved to \SL{} by \cite{bm89}.
In \cite{bklm01}, the group membership problem for several restricted classes of groups was placed in \FOLL, a class similar to $\AC^0$ but with $O(\log \log n)$ depth.
The more recent proof that $\SL = \L$ \cite{reingold08} places the membership problem for general groups (and thus for restricted classes of groups) squarely in \L.

Also related is the problem of deciding whether two given groups are isomorphic.
A series of works improved the upper bound for the \emph{quasi}group isomorphism problem (and hence the group isomorphism problem) from $\GC(\log^2 n, \P)$ \cite{py96} to $\GC(\log^2 n, \NC^2)$ \cite{wolf94} to $\GC(\log^2 n, \SAC^1)$ \cite{wagner10} to $\GC(\log^2 n, \FOLL)$ \cite{ctw10}.

\section{Preliminaries}\label{sec:prelim}

\P{} is the class of decision problems decidable by a deterministic Turing machine halting in polynomial time.
$\L^2$ is the class of decision problems decidable by a deterministic Turing machine using at most $O(\log^2 n)$ space.
The (inclusion) relationship between \P{} and $\L^2$ is unknown.
\FOLL{} is the class of decision problems decidable by a $\DTIME(\log n)$-uniform family of circuits with polynomial size, unbounded fan-in, and $O(\log \log n)$ depth.
$\FOLL^2$ is the same, but with $O(\log^2 \log n)$.
Following the notation of Cai and Chen \cite{cc97}, $\GC(f(n), \mathcal{C})$ is the class of decision problems decidable by a $\mathcal{C}$ machine augmented with $O(f(n))$ nondeterministic bits.
We will consider specifically the complexity classes $\GC(\log^2 n, \L)$ and $\GC(\log^2 n, \FOLL)$.
%% TODO definitions for many-one and conjunctive truth table reductions

If $(G, \cdot)$ is a finite group of order $n$, we identify the elements of the group with the indices 1, 2, \ldots, n, where 1 is the identity of the group.
We sometimes refer to the group as $(G, \cdot)$, but will also sometimes omit the group product and just write $G$.
The \emph{Cayley table} of this group is the $n \times n$ matrix with entries from $G$ in which entry $(u, v)$ has value $g$ if $u \cdot g = v$.
We will consider all elements of $G$ to be expressed in binary, so the size of each element of $G$ is $\log n$ and the size of the Cayley table is $n^2 \log n$.
If $T \subseteq G$, the set \emph{generated} by $T$, denoted $\gen{T}$, is the transitive closure of $\cdot$ on $T$.
A \emph{generating set} for $(G, \cdot)$ is a set $T \subseteq G$ such that $\gen{T} = G$.

We will also consider subclasses of the class of all finite groups: \textsc{Group}, \textsc{Constant Solvability Class}, \textsc{Nilpotent}, \textsc{Abelian}, \textsc{p-Group}, \textsc{Elementary Abelian}, and \textsc{Cyclic}, where \textsc{Group} is the class of all groups, \textsc{Constant Solvability Class} is the class of all groups of solvability class $O(1)$, etc.
From group theory we know that for finite groups,
\begin{equation*}
  \textsc{Cyclic} \subseteq \textsc{Elementary Abelian} \subseteq \textsc{Abelian} \subseteq \textsc{Nilpotent} \subseteq \textsc{Group},
\end{equation*}
\begin{equation*}
  \textsc{p-Group} \subseteq \textsc{Nilpotent} \subseteq \textsc{Group},
\end{equation*}
and
\begin{equation*}
  \textsc{Constant Solvability Class} \subseteq \textsc{Group}.
\end{equation*}

\section{Decision problem for computing the size of the minimum generating set}

In order to study the complexity of computing the size of the minimum generating set, we define a decision problem corresponding to this optimization problem.

\begin{definition}[\textsc{Min Gen Size($\mathcal{G}$)}]
  \mbox{}

  \textbf{Instance:} finite group $(G, \cdot)$ (given as a Cayley table) in the class $\mathcal{G}$, natural number $k$ (given in binary).

  \textbf{Question:} Does there exist a $T \subseteq G$ such that $\gen{T} = G$ and $|T| \leq k$?
\end{definition}

Since each of the other classes of groups is a subclass of \textsc{Group}, an upper bound on the complexity of \textsc{Min Gen Size(Group)} is also an upper bound on the complexity of \textsc{Min Gen Size} for all the other subclasses.

%%%%%%%%%%%%%%%%%%%%%%%%%%%%%%%%%%%%%%%%%%%%%%%%%%%%%%%%%%%%%%%%%%%%%%%%%%%%%%%
%% TODO this doesn't work because according to Wikipedia, a nontrivial finite %
%% group is not a free group.                                                 %
%%%%%%%%%%%%%%%%%%%%%%%%%%%%%%%%%%%%%%%%%%%%%%%%%%%%%%%%%%%%%%%%%%%%%%%%%%%%%%%
%% We will show that the upper bound for \textsc{Min Gen Size(Group)} is \P.
%% In order to show that the problem is decidable in deterministic polynomial time, we reduce it to the \textsc{Min Rank(Free Group}} problem, which is known to be \P-complete \cite[Problem~A.8.11]{ghr95}.

%% \begin{definition}[\textsc{Min Rank(Free Group)}]
%%   \mbox{}

%%   \textbf{Instance:} finite set $S$, finite set $U \subseteq (S \cup S^{-1})^*$, natural number $k$ (given in binary).

%%   \textbf{Question:} Does there exist a $T \subseteq \gen{U}$ such that $\gen{T} = \gen{U}$ and $|T| \leq k$?
%% \end{definition}

%% \begin{lemma}\label{lem:mingenp}
%%   $\textsc{Min Gen Size(Group)} \in \P$.
%% \end{lemma}
%% \begin{proof}
%%   We show a logarithmic space many-one reduction from \textsc{Min Gen Size(Group)} to \textsc{Min Rank(Free Group)}.
%%   Since \textsc{Min Rank(Free Group)} is in \P{} and \P{} is closed under polynomial time many-one reductions, we conclude that \textsc{Min Rank(Free Group)} is also in \P.
%%   The reduction is the mapping $\langle (G, \cdot), k \rangle \mapsto \langle G, G, k \rangle$.
%%   First, we observe that $\langle G, G, k \rangle$ is a correctly formatted instance of the reduced problem: $G$ is a finite set and $G \subseteq (G \cup G^{-1})^*$.

%%   Suppose that $\langle (G, \cdot), k \rangle \in \textsc{Min Gen Size(Group)}$, so there exists a $T \subseteq G$ such that $\gen{T} = G$ and $|T| \leq k$.
%%   Since $G$ is closed under $\cdot$, we know $\gen{G} = G$ and hence $\gen{T} = \gen{G}$.
%%   Therefore we conclude that $\langle G, G, k \rangle \in \textsc{Min Rank(Free Group)}$.
%%   For the converse, suppose that $\langle G, G, k \rangle \in \textsc{Min Rank(Free Group)}$, so there exists a $T \subseteq \gen{G}$ such that $\gen{T} = \gen{G}$.
%%   As stated above, $\gen{G} = G$, so $\gen{T} = G$.
%%   Therefore $\langle (G, \cdot), k \rangle \in \textsc{Min Gen Size(Group)}$.

%%   We have shown a correct logarithmic space many-one reduction to a problem in \P, and we conclude that $\textsc{Min Gen Size(Group)} \in \P$.
%% \end{proof}

Previously, the \textsc{Min Gen Size(Group)} problem was known to be in $\L^2$, the class of problems decidable by a deterministic Turing machine using at most $O(\log^2 n)$ space \cite{lsz77} (see \cite[Proposition~3]{at06} for a brief description of the algorithm; I can't find a copy of \cite{lsz77} online).
We can improve this upper bound by a more careful analysis of the $\L^2$ algorithm.
We require a definition of the following auxiliary problem.

\begin{definition}[\textsc{Membership($\mathcal{G}$)}]
  \mbox{}

  \textbf{Instance:} finite group $(G, \cdot)$ (given as a Cayley table) in the class $\mathcal{G}$, finite set $S \subset G$, group element $v \in G$.

  \textbf{Question:} Is $v \in \gen{X}$?
\end{definition}

\begin{lemma}\label{lem:membershipinl}
  $\textsc{Membership(Group)} \in \L$.
\end{lemma}
\begin{proof}
  Since $\textsc{Membership(Group)} \in \SL$ \cite[Section~3]{bm89}, and $\SL = \L$ \cite{reingold08}, the lemma follows.
\end{proof}

Upper bounds on the complexity of \textsc{Membership($\mathcal{G}$)} for several subclasses of \textsc{Group} are explored in \cite{bklm01}.
%We use the logarithmic space algorithm for this problem to give a simple $\GC(\log^2 n, \L)$ algorithm for \textsc{Min Group Gen}.

Before providing the general algorithms for computing the size of the minimum generating set for classes of groups, we require one algebraic lemma which bounds the size of the minimum generating set of any finite group.

\begin{lemma}\label{lem:log}
  If $(G, \cdot)$ is a finite group then the size of the minimum generating set is at most $\log n$.
\end{lemma}
\begin{proof}
  Proof can be found at \url{http://math.stackexchange.com/a/226938/29369}, add it here.
\end{proof}

This lemma allows us to consider, without loss of generality, only inputs $\langle (G, \cdot), k \rangle$ such that $k \leq \log n$.

\subsection{In logarithmic space with bounded nondeterminism}

\begin{theorem}\label{thm:mingengc}
  $\textsc{Min Gen Size(Group)} \in \GC(\log^2 n, \L)$.
\end{theorem}
\begin{proof}
  The algorithm proceeds as follows on input $\langle (G, \cdot), k \rangle$, where $G$ is a group of order $n$:
  \begin{itemize}
  \item Nondeterministically guess a subset $S \subseteq G$ of cardinality at most $k$.
  \item Accept if and only if for all $v \in G$, $\langle G, S, v \rangle \in \textsc{Membership(Group)}$.
  \end{itemize}

  First we prove that this algorithm uses $O(\log^2 n)$ nondeterministic bits and $O(\log n)$ space.
  The size of each group element, represented as number in binary, is $\log n$.
  By \autoref{lem:log}, we assume without loss of generality that the size of the minimum generating set for a group of order $n$ is at most $\log n$, so $k$ must be at most $\log n$.
  Therefore the total number of nondeterministic bits required to guess $S$ is $O(\log^2 n)$.
  Iterating over all elements of $G$ requires $O(\log n)$ space to keep track of the current element in the iteration.
  Since $\textsc{Membership(Group)} \in \L$ by \autoref{lem:membershipinl}, deciding whether $\langle G, S, v \rangle \in \textsc{Membership(Group)}$ uses at most $O(\log n)$ space.
  Therefore the total space required for this algorithm (other than the read-only input and read-only nondeterministic bits) is $O(\log n)$.

  Next we show that the algorithm correctly decides the problem.
  Suppose $\langle (G, \cdot), k) \rangle \in \textsc{Min Gen Size(Group)}$, so there exists a $T \subseteq G$ such that $|T| \leq k$ and $\gen{T} = G$.
  One of the sets $S$ of cardinality at most $k$ that the algorithm guesses will equal $T$, since $|T| \leq k$.
  For all elements $v \in G$, we have $v \in \gen{T}$ since $\gen{T} = G$.
  Hence the (correct) algorithm for \textsc{Membership(Group)} will accept for all $v \in G$, and the overall algorithm will accept.
  For the converse, suppose the algorithm accepts the input $\langle (G, \cdot), k \rangle$.
  This occurs exactly when it has guessed a set $S$ of cardinality at most $k$ such that all elements $v$ in $G$ are members of the subgroup generated by $S$.
  Thus $S$ is a generating set for $G$ of cardinality at most $k$, so $\langle (G, \cdot), k \in \textsc{Min Gen Size(Group)}$.
\end{proof}

This improves the previous upper bound, $\L^2$, because
\begin{equation*}
  \GC(\log^2 n, \L) \subseteq \GC(\log^2 n, \NC^2) \subseteq \L^2,
\end{equation*}
where the first inclusion follows from the fact that $\L \subseteq \NL \subseteq \NC^2$, and the second is proven in \cite[Lemma~3.2.8]{wolf90}.

%% From \autoref{lem:mingenp} and \autoref{lem:mingengc}, we immediately get the following theorem.

%% \begin{theorem}
%%   $\textsc{Min Group Gen} \in \GC(\log^2 n, \L) \cap \P$.
%% \end{theorem}

%% We conjecture that this is not the best possible upper bound, and more clever algebra may reveal a more efficient algorithm.

%% \begin{conjecture}
%%   $\textsc{Min Gen Size(Group)} \in \L$.
%% \end{conjecture}

\subsection{In \texorpdfstring{\FOLL}{FOLL} with bounded nondeterminism}

%% TODO The size of the minimum generating set for cyclic groups is 1.
\begin{theorem}
  \mbox{}
  \begin{enumerate}
  \item Each of the problems
    \begin{itemize}
    \item \textsc{Min Gen Size(Cyclic)},
    \item \textsc{Min Gen Size(Elementary Abelian)},
    \item \textsc{Min Gen Size(Abelian)}, and
    \item \textsc{Min Gen Size(Constant Solvability Class)}
    \end{itemize}
    is in $\GC(\log^2 n, \FOLL)$.
  \item $\textsc{Min Gen Size(Nilpotent)} \in \GC(\log^2 n, \FOLL^2)$.
  \end{enumerate}
\end{theorem}
\begin{proof}
  \mbox{}
  \begin{enumerate}
  \item
    We know that the membership problem for each of these classes of groups is in \FOLL{} by \cite[Section~3]{bklm01}.
    A slight modification of the algorithm from \autoref{thm:mingengc} can be used to show that each of these problems is in $\GC(\log^2 n, \FOLL)$.
    After nondeterministically guessing a set $S$, instead of iterating over each element of the group in logarithmic space, we use $n$ instances of the \FOLL{} circuit which decides the membership problem and compute the conjunction of the output of each of the instances for one additional layer of depth and a $O(n)$ increase in the size of the circuit.
    Since the size of the circuit remains polynomial in $n$ and the depth remains $O(\log \log n)$, we have shown $\textsc{Min Gen Size(Cyclic)} \in \GC(\log^2 n, \FOLL)$.
  \item The same argument applies, except using the fact that the membership problem for nilpotent groups is in $\FOLL^2$ \cite[Corollary~3.12]{bklm01}. \qedhere
  \end{enumerate}
\end{proof}
%% Observe that even though, for example, $\textsc{Membership(Cyclic)} \in \L \cap \FOLL$, we show that \textsc{Min Gen Size(Cyclic)} is in
%% \begin{equation*}
%%   \GC(\log^2 n, \L) \cap \GC(\log^2 n, \FOLL)
%% \end{equation*}
%% instead of
%% \begin{equation*}
%%   \GC(\log^2 n, \L \cap \FOLL),
%% \end{equation*}
%% for syntactic reasons (the second step in the algorithm from \autoref{thm:mingengc} is either a logarithmic space Turing machine procedure or a $\FO(\log \log n)$ procedure, but not both simultaneously).

\subsection{In polynomial time}

At this point, we provide a clearer presentation of the proof of \cite[Theorem~7]{at06}.
We have split up the proof into multiple parts in order to more easily analyze their complexity.
We start with the algebraic lemmata, then provide the necessary reductions.

\begin{lemma}\label{lem:sylow}
  Suppose $(G, \cdot)$ is a finite group of order $n$, where the (unique) prime factorization of $n$ is $\Pi_{i = 1}^m p_i^{e_i}$.
  Then $G$ is nilpotent if and only if $G \cong S_{p_1} \times S_{p_2} \times \dotsb \times S_{p_m}$ where each $S_{p_i}$ is the unique $p_i$-Sylow subgroup of $G$.
\end{lemma}

%% \begin{lemma}
%%   Suppose $(G, \cdot)$ and $(H, \circ)$ are groups such that $G \cong H$ and $k$ is a non-negative integer.
%%   $G$ has a generating set of size $k$ if and only if $H$ has a generating set of size $k$.
%% \end{lemma}

\begin{proposition}
  \begin{equation*}
    \textsc{Min Gen Size(Nilpotent)} \leq_{ctt}^P \textsc{Min Gen Size(p-Group)}.
  \end{equation*}
\end{proposition}
\begin{proof}
  Define the reduction by
  \begin{equation*}
    \langle (G, \cdot), k \rangle \mapsto \bigwedge_{i=1}^m \langle (S_{p_i}, \cdot), k \rangle
  \end{equation*}
  for all nilpotent groups $(G, \cdot)$ of order $n$ and all positive integers $k$, where the prime factorization of $n$ is $p_1^{e_1}p_2^{e_2}\dotsb p_m^{e_m}$ and $S_{p_i}$ is the unique $p_i$-Sylow subgroup of $G$.
  Observe that the output is correctly formatted because each of the Sylow subgroups is a $p$-group.

  First we show that this reduction is computable in deterministic polynomial time.
  Since the length of the input is $n^2 \log n$, the prime factorization of $n$ can be computed in deterministic polynomial time with respect to the length of the input (using, for example, the general number field sieve \cite{llmp93}).
  We know that the set of Sylow subgroups exists by \autoref{lem:sylow}.
  If $n_i = n p_i^{-e_i}$ then $S_{p_i} = \{g^{n_i} \, | \, g \in G\}$ for all $i \leq m$.
  Computing the $n_i$th power of $g$ can be computed in logarithmic space (by iteratively computing $g$, $g^2$, $g^3$, etc.)
  Computing the set $S_{p_i}$ can therefore be computed in logarithmic space as well.
  Constructing the Cayley table of $S_{p_i}$ can be performed in constant space by simply looking up the appropriate entries in the Cayley table of the group $G$.
  The total number of clauses in the conjunction is $m$, which is the number of prime factors of $n$, which is bounded by a polynomial in $n^2 \log n$ because the number of prime factors of $n$ is less than $n$.
  Hence the reduction is polynomial time computable.

  Next we show that the reduction is correct.
  Suppose first that $\langle (G, \cdot), k \rangle \in \textsc{Min Gen Size(Nilpotent)}$, so there exists a subset of $G$, say $\{g_1, g_2, \dotsc, g_k\}$, such that $\gen{g_1, g_2, \dotsc, g_k} = G$.
  If $n_i = n p_i^{-e_i}$, as defined above, then the set $\{g_1^{n_i}, g_2^{n_i}, \dotsc, g_k^{n_i} \}$ is a generating set for $S_{p_i}$.
  So for each $i$ with $1 \leq i \leq m$, we have $\langle (S_{p_i}, \cdot), k \rangle \in \textsc{Min Gen Size(p-Group)}$.
  For the converse, suppose that each $S_{p_i}$ has a generating set of size $k$, say $\{g_{i, 1}, g_{i, 2}, \dotsc, g_{i, k}\}$.
  Let $g_j = \Pi_{i = 1}^m g_{i, j}$ for all $j$ with $1 \leq j \leq k$.
  In other words, $g_j$ is the product of all the $j$th elements in the generating sets of $S_{p_1}$, $S_{p_2}, \dotsc, S_{p_m}$.
  Since $\gen{g_1^{n_i}, g_2^{n_i}, \dotsc, g_k^{n_i}} = S_{p_i} $, we have $\gen{g_1, g_2, \dotsc, g_k} = G$.

  We have now shown a correct deterministic polynomial time conjunctive truth table reduction from \textsc{Min Gen Size(Nilpotent)} to \textsc{Min Gen Size(p-Group)}.
\end{proof}

\begin{lemma}\label{lem:decompose}
  If $(G, \cdot)$ is a finite abelian group, then there exists a finite set of cyclic groups $\{H_1, H_2, \dotsc, H_t\}$ such that $G \cong H_1 \times H_2 \times \dotsb \times H_t$.
\end{lemma}

\begin{lemma}\label{lem:factorgroup}
  If $(G, \cdot)$ is a group of order $n$ given by its Cayley table and $H < G$ then the function $\langle (G, \cdot), H \rangle \mapsto G / H$ is computable in polynomial time.
\end{lemma}
\begin{proof}
  The elements of $G / H$ can be computed by iteratively computing $g H$ for each $g \in G$.
  There are $O(n)$ such iterations, each of which requires $O(n)$ multiplications, where each multiplication requires constant time (by lookup in the Cayley table).
  Constructing the table \ldots
\end{proof}

\begin{definition}
  Suppose $(G, \cdot)$ is a group.
  The \emph{Frattini subgroup} of $G$, denoted $\Phi(G)$, is the intersection of all maximal subgroups of $G$.
\end{definition}

\begin{lemma}\label{lem:frattinip}
  Let $(G, \cdot)$ be a finite $p$-group.
  Then the function $G \mapsto \Phi(G)$ is computable in polynomial time.
\end{lemma}
\begin{proof}
  Let $G'$ be the commutator subgroup of $G$.
  We know $G' \triangleleft G$ and $G/G'$ is abelian.
  $G$ is nilpotent if and only if $G' \subseteq \Phi(G)$.
  Since $G$ is a finite $p$-group, $G$ is nilpotent.
  For any subgroup $N < G$, if $N \triangleleft G$ and $N \subseteq \Phi(G)$ then $\Phi(G / N) \cong \Phi(G) / N$.
  Hence $\Phi(G / G') \cong \Phi(G) / G'$.
  If the Cayley tables for $\Phi(G / G')$ and $G'$ can be computed in polynomial time, then so can the Cayley table for $\Phi(G)$ using this isomorphism.
  Since the table for $G'$ can be computed in polynomial time, it therefore suffices to show that the table for $\Phi(G / G')$ is computable in polynomial time.

  Since the table for $G'$ can be computed in polynomial time, so can the table for $G / G'$.
  By \autoref{lem:decompose} we can compute in polynomial time the decomposition of $G / G'$ into a product of cyclic groups $\{H_1 / G', H_2 / G', \dotsc, H_t / G'\}$, so $G / G' \cong H_1 / G' \times H_2 / G' \times \dotsb \times H_t / G'$.
  For each $j$ with $1 \leq j \leq t$, let $y_jG'$ be the element in $G / G'$ which generates $H_j / G'$, and suppose the order of $y_jG'$ is $p^{c_j}$ for some non-negative integer $c_j$.
  We know the order of $y_jG'$ must be a power of $p$ because the order of $G$ is a prime power.
  Since the group is given as a Cayley table, we can compute the generator of a cyclic group in polynomial time by iterating over each element and testing whether it is a generator.

  We know $\Phi(G / G') \cong \Phi(H_1 / G') \times \Phi(H_2 / G') \times \dotsb \times \Phi(H_t / G')$.
  Since each factor group $H_j / G'$ is a cyclic group of prime power order, it has a unique maximal proper subgroup, specifically the subgroup $\gen{y_j^pG'}$.
  Therefore, by definition of the Frattini subgroup, $\Phi(H_j / G') = \gen{y_j^pG'}$ for all $j$ with $ 1 \leq j \leq t$.
  Working our way back through the isomorphisms, we find that $\Phi(G / G') = \gen{y_1^pG', y_2^pG', \dotsc, y_t^pG'}$, and hence $\Phi(G) = \gen{y_1^p, y_2^p, \dotsc, y_t^p, G'}$.
  Since we can compute each generator $y_j$ in polynomial time, since we can compute $y_j^p$ in polynomial time, and since $t$ is bounded by $O(n)$, the size of $\Phi(G)$ is polynomial in $n$.
\end{proof}

\begin{proposition}
  \begin{equation*}
    \textsc{Min Gen Size(p-Group)} \leq_m^P \textsc{Min Gen Size(Elementary Abelian)}.
  \end{equation*}
\end{proposition}
\begin{proof}
  Define the reduction by
  \begin{equation*}
    \langle (G, \cdot), k \rangle \mapsto \langle (G / \Phi(G), \cdot), k \rangle
  \end{equation*}
  for all $p$-groups $G$ of order $n$ and all positive integers $k$, where $\Phi(G)$ is the Frattini subgroup of $G$.
  By \autoref{lem:frattinip}, the Cayley table of $\Phi(G)$ is computable in polynomial time.
  By \autoref{lem:factorgroup}, the Cayley table of $G / \Phi(G)$ is computable in polynomial time.
  Hence the reduction is polynomial time computable.

  Suppose $\langle (G, \cdot), k \rangle \in \textsc{Min Gen Size(p-Group)}$, so $G$ has a generating set of size at most $k$, say $\{g_1, g_2, \dotsc, g_k\}$.
  Then $\{g_1\Phi(G), g_2\Phi(G), \dotsc, g_k\Phi(G)\}$ is a generating set for $G / \Phi(G)$ of size at most $k$.
  For the converse suppose $\langle (G / \Phi(G), \cdot), k \rangle \in \textsc{Min Gen Size(Elementary Abelian)}$, so $G / \Phi(G)$ has a generating set of size at most $k$, say $\{g_1\Phi(G), g_2\Phi(G), \dotsc, g_k\Phi(G)\}$.
  If we let $H = \gen{g_1, g_2, \dotsc, g_k}$, then $H \Phi(G) = G$.
  Since $\Phi(G)$ is a set containing only nongenerators for $G$, we have $H = G$.
  Therefore $G$ has a generating set of size at most $k$.
  This concludes the proof.
\end{proof}

\begin{lemma}
  $\textsc{Min Gen Size(Elementary Abelian)} \in \P$.
\end{lemma}
\begin{proof}
  Every elementary abelian group is isomorphic to $(\mathbb{Z} / p\mathbb{Z})^m$ where $p$ is a prime number and $m$ is a non-negative integer.
  The minimum generating set for $\mathbb{Z} / p\mathbb{Z}$ is $\{1\}$, so the minimum generating set for $(\mathbb{Z} / p\mathbb{Z})^m$ is
  \[
  \{(1, 0, \dotsc, 0), (0, 1, 0, \dotsc, 0), \dotsc, (0, \dotsc, 0, 1)\}.
  \]
  Therefore the size of the minimum generating set for every elementary abelian group is $m$.

  On input $\langle (G, \cdot), k \rangle$, the polynomial time algorithm computes $m$, then compares that with $k$.
  To compute $m$, determine the order of the group $n$ and and compute its factorization, which can be done in polynomial time when the input is given as a Cayley table (using, for example, the general number field sieve \cite{llmp93}).
\end{proof}

\begin{theorem}
  $\textsc{Min Gen Size(Nilpotent)} \in \P$.
\end{theorem}
\begin{proof}
  Since a many-one reduction implies a conjunctive truth table reduction, we have
  \begin{equation*}
    \textsc{Min Gen Size(Nilpotent)} \leq_{ctt}^P \textsc{Min Gen Size(p-Group)},
  \end{equation*}
  and
  \begin{equation*}
    \textsc{Min Gen Size(p-Group)} \leq_{ctt}^P \textsc{Min Gen Size(Elementary Abelian)}.
  \end{equation*}
  Since polynomial time conjunctive truth table reductions compose this implies
  \begin{equation*}
    \textsc{Min Gen Size(Nilpotent)} \leq_{ctt}^P \textsc{Min Gen Size(Elementary Abelian)}.
  \end{equation*}
  Since $\textsc{Min Gen Size(Elementary Abelian)} \in \P$ and $\P$ is closed under polynomial time conjunctive truth table reductions, $\textsc{Min Gen Size(Nilpotent)} \in \P$.
\end{proof}

\section{Searching for the exact size of the minimum generating set}

Now that we know the complexity for the decision problem form of \textsc{Min Gen Size(Group)}, we wish to implement a function which computes the exact size of the minimum generating set of a group.
For decision problems in polynomial time, the well-known binary search on $k$ between 1 and $n$, where $n$ is the order of the group, places the search problem in polynomial time as well.
For decision problems in $\GC(\log^2 n, \L)$, however, each level of the binary search recursion tree requires an additional $O(\log^2 n)$ bits of nondeterminism.
(The $O(\log n)$ space can still be reused at each level.)
For this reason, the search problem corresponding to \textsc{Min Gen Size(Group)} is computable in $\GC(\log^3 n, \L)$.

TODO how about binary search in $\GC(\log^2 n, \FOLL)$?

\section{Completeness}

TODO mention things in GC(log2 n, FOLL) cant be hard for small classes under AC0 reductions.

\section{Future work}

\autoref{fig:results} contains a summary of the results found in this work.

%% TODO ensure that this is correct
\begin{figure}
\caption{A summary of the upper bounds due to this work for the problem of computing the size of the minimum generating set of a class of groups.\label{fig:results}}
  \begin{center}
    \begin{tabular}{l | l}
      \multicolumn{1}{c |}{Subclass $\mathcal{G}$ of \textsc{Group}}
      &
      \multicolumn{1}{| c}{\textsc{Min Gen Size($\mathcal{G}$)} decision problem} \\
      \hline
      \hline
      \textsc{Group} & $\GC(\log^2 n, \L)$ \\
      \textsc{Solvable} & $\GC(\log^2 n, \L)$ \\
      \textsc{Nilpotent} & $\GC(\log^2 n, \L) \cap \GC(\log^2 n, \FOLL^2) \cap \P$ \\
      \textsc{p-Group} & $\GC(\log^2 n, \L) \cap \GC(\log^2 n, \FOLL^2) \cap \P$ \\
      \textsc{Constant Solvability Class} & $\GC(\log^2 n, \L) \cap \GC(\log^2 n, \FOLL)$ \\
      \textsc{Abelian} & $\GC(\log^2 n, \L) \cap \GC(\log^2 n, \FOLL) \cap \P$ \\
      \textsc{Elementary Abelian} & $\GC(\log^2 n, \L) \cap \GC(\log^2 n, \FOLL) \cap \P$ \\
      \textsc{Cyclic} & $\GC(\log^2 n, \L) \cap \GC(\log^2 n, \FOLL) \cap \P$ \\
      \multicolumn{2}{c}{} \\[10pt]
      \multicolumn{1}{c |}{Subclass $\mathcal{G}$ of \textsc{Group}}
      &
      \multicolumn{1}{| c}{\textsc{Min Gen Size($\mathcal{G}$)} search problem} \\
      \hline
      \hline
      \textsc{Group} & $\GC(\log^3 n, \L)$ \\
      \textsc{Solvable} & $\GC(\log^3 n, \L)$ \\
      \textsc{Nilpotent} & $\GC(\log^3 n, \L) \cap \P$ \\
      \textsc{p-Group} & $\GC(\log^3 n, \L) \cap \P$ \\
      \textsc{Constant Solvability Class} & $\GC(\log^3 n, \L)$ \\
      \textsc{Abelian} & $\GC(\log^3 n, \L) \cap \P$ \\
      \textsc{Elementary Abelian} & $\GC(\log^3 n, \L) \cap \P$ \\
      \textsc{Cyclic} & $\GC(\log^3 n, \L) \cap \P$
    \end{tabular}
  \end{center}
\end{figure}

As noted in the introduction, the (quasi)group isomorphism problem is known to be in $\GC(\log^2 n, \FOLL)$.
What is the relationship between this problem and the minimum generating set size problem for groups?

What is the relationship between computing the size of the minimum generating set and computing the minimum generating set itself?

\section{About this work}

Copyright 2012 Jef{}frey Finkelstein.

This work is licensed under the Creative Commons Attribution-ShareAlike License 3.0.
Visit \mbox{\url{https://creativecommons.org/licenses/by-sa/3.0/}} to view a copy of this license.

The \LaTeX{} markup which generated this document is available on the World Wide Web at \mbox{\url{https://github.com/jfinkels/ncapproximation}}.
It is also licensed under the Creative Commons Attribution-ShareAlike License.

The author can be contacted via email at \email{jeffreyf@bu.edu}.

\bibliographystyle{plain}
\bibliography{references}

\end{document}
