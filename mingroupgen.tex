\documentclass[]{article}

% Package `amsthm` and `thmtools` must come before package `hyperref`.
\usepackage{amsthm}
\usepackage{thmtools}
% Package `hyperref` must come before package `complexity`.
\usepackage[pdftitle={Restricted probabilistically checkable proofs}, pdfauthor={Jeffrey Finkelstein}]{hyperref}
\usepackage{complexity}
\usepackage{amsmath}

\newcommand{\gen}[1]{{\langle #1 \rangle}}

\declaretheorem[]{theorem}
%\declaretheorem[numberlike=theorem]{corollary}
%\declaretheorem[numberlike=theorem]{lemma}
%\declaretheorem[numberlike=theorem]{proposition}
\declaretheorem[numberlike=theorem, style=definition]{definition}

\title{Complexity of computing minimum generating sets for groups}
\author{Jef{}frey~Finkelstein}
\date{\today}

\begin{document}
\maketitle

%% A \emph{group} is a two-tuple $(G, \cdot)$ where $G$ is a (finite or infinite) set and $\cdot$ is a binary operation on $G$ satisfying closure, associativity, and invertibility.
%% The \emph{Cayley table} of a group is a matrix whose rows and columns are labelled by elements of $G$, and where entry $x$ is labelled by $(a, b)$ if $a \cdot b = x$.

If $(G, \cdot)$ is a group and $S$ is a subset of $G$ then the set of words on $S$ is denoted $S^*$ and the set of inverses of elements in $S$ is denoted $S^{-1}$.
If $T \subseteq G$, the set \emph{generated} by $T$, denoted $\gen{T}$, is defined by
\begin{equation*}
  \gen{T} = \{ x \in G | x \text{ is the product of elements in } T \}.
\end{equation*}
A \emph{generating set} for $(G, \cdot)$ is a set $T \subseteq G$ such that $\gen{T} = G$.


\begin{definition}[\textsc{Minimum Group Generating Set}]
  \mbox{}

  \textbf{Instance:} finite group $(G, \cdot)$ (given as a Cayley table).

  \textbf{Solution:} finite set $T$ such that $\gen{T} = G$.

  \textbf{Measure:} $|T|$.

  \textbf{Type:} $\min$.
\end{definition}

\begin{definition}[\textsc{Minimum Group Generating Set Budget}]
  \mbox{}

  \textbf{Instance:} finite group $(G, \cdot)$ (given as a Cayley table), natural number $k$.

  \textbf{Question:} Does there exist a $T \subseteq G$ such that $\gen{T} = G$ and $|T| \leq k$?
\end{definition}

\begin{definition}[\textsc{Minimum Group Rank}]
  \mbox{}

  \textbf{Instance:} finite set $S$, finite set $U \subseteq (S \cup S^{-1})^*$.

  \textbf{Solution:} finite set $T$ such that $\gen{T} = \gen{U}$.

  \textbf{Measure:} $|T|$.

  \textbf{Type:} $\min$.
\end{definition}

\begin{definition}[\textsc{Minimum Group Rank Budget}]
  \mbox{}

  \textbf{Instance:} finite set $S$, finite set $U \subseteq (S \cup S^{-1})^*$, natural number $k$.

  \textbf{Question:} Does there exist a $T \subseteq \gen{U}$ such that $\gen{T} = \gen{U}$ and $|T| \leq k$?
\end{definition}

Previously, the \textsc{Minimum Group Generating Set Budget} problem was known to be in $\L^2$, the class of problems decidable by a deterministic Turing machine using at most $O(\log^2 n)$ space \cite{lsz77}.

\begin{theorem}
  $\textsc{Minimum Group Generating Set Budget} \in \P \cap \L^2$.
\end{theorem}
\begin{proof}
  We show a logarithmic space many-one reduction from \textsc{Minimum Group Generating Set Budget} to \textsc{Minimum Group Rank Budget}.
  Since \textsc{Minimum Group Rank Budget} is in \P{} (in fact, it is complete for \P{} under logarithmic space many-one reductions \cite{ghr95}) and \P{} is closed under polynomial time many-one reductions, we conclude that \textsc{Minimum Group Generating Set Budget} is also in \P.
  The reduction is the mapping $\langle (G, \cdot), k \rangle \mapsto \langle G, G, k \rangle$.
  First, we observe that $\langle G, G, k \rangle$ is a correctly formatted instance of the reduced problem: $G$ is a finite set and $G \subseteq (G \cup G^{-1})^*$.

  Suppose that $\langle (G, \cdot), k \rangle \in \textsc{Minimum Group Generating Set Budget}$, so there exists a $T \subseteq G$ such that $\gen{T} = G$ and $|T| \leq k$.
  Since $G$ is closed under $\cdot$, we know $\gen{G} = G$ and hence $\gen{T} = \gen{G}$.
  Therefore we conclude that $\langle G, G, k \rangle \in \textsc{Minimum Group Rank Budget}$.
  For the converse, suppose that $\langle G, G, k \rangle \in \textsc{Minimum Group Rank Budget}$, so there exists a $T \subseteq \gen{G}$ such that $\gen{T} = \gen{G}$.
  As stated above, $\gen{G} = G$, so $\gen{T} = G$.
  Therefore $\langle (G, \cdot), k \rangle \in \textsc{Minimum Group Generating Set Budget}$.

  We have shown a correct logarithmic space many-one reduction to a problem in \P, and we conclude that $\textsc{Minimum Group Generating Set Budget} \in \P$.
\end{proof}

\bibliographystyle{plain}
\bibliography{references}

\end{document}
