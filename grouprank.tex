\documentclass{article}

\usepackage{amsmath}
% This must come before hyperref.
\usepackage{amsthm}
% This must come before hyperref.
\usepackage{thmtools}
% This must come before complexity.
\usepackage[pdftitle={Computing Group Rank with Limited Nondeterminism}, pdfauthor={Jeffrey Finkelstein}]{hyperref}
%% This is strongly recommended by biblatex.
\usepackage[english]{babel}
%% This is strongly recommended by biblatex.
\usepackage{csquotes}
\usepackage[backend=biber]{biblatex}
\usepackage{complexity}

\addbibresource{references.bib}

\declaretheorem[]{theorem}
\declaretheorem[numberlike=theorem]{corollary}

\newcommand{\email}[1]{\href{mailto:#1}{\nolinkurl{#1}}}

\title{Computing Group Rank with Limited Nondeterminism}
\author{Jeffrey~Finkelstein}
%\date{\today}

\begin{document}

\maketitle

\L{} is the class of languages decidable by a deterministic Turing machine that uses $O(\log n)$ space on inputs of length $n$.
%$\L^2$ is the class of languages decidable in $O(\log^2 n)$ space.
\NL{} is the class of languages decidable by a nondeterministic Turing machine using $O(\log^2 n)$ space.
$\bL$ is the subclass of \NL{} in which the nondeterministic Turing machine uses at most $O(\log^2 n)$ nondeterministic bits.
\FOLL{} is the class of languages decidable by a \L-uniform family of circuits with polynomial size, unbounded fan-in, and $O(\log \log n)$ depth.
$\bFOLL$ is the class of languages decidable by \FOLL{} circuits that have been augmented with $O(\log^2 n)$ nondeterministic bits (gates with no inputs and one output).
%%$\NC^2$ is the class of languages decidable by a \L-uniform family of circuits with polynomial size, fan-in 2, and $O(\log^2 n)$ depth.

A \emph{quasigroup} is a set $G$ with a binary operation $\cdot$ such that for each $a$ and $b$ in $G$ there exist unique elements $x$ and $y$ in $G$ such that $a \cdot x = b$ and $y \cdot a = b$.
(In other words, each quasigroup element appears exactly once in each row and each column of the multiplication table of $G$.)
If the quasigroup is nonempty and associative, then it is a \emph{group}
A \emph{parenthesization} $P$ of a sequence of quasigroup elements $(g_0, \dotsc, g_k)$ is a binary tree that has the quasigroup elements as its leaves (in the order indicated by the sequence).
The \emph{parenthesized product} of a sequence of quasigroup elements $(g_0, \dotsc, g_k)$ with parenthesization $P$, denoted $P(g_0, \dotsc, g_k)$, is the quasigroup element that results from performing the quasigroup product in the order indicated by the parenthesization.
A \emph{generating sequence of size $k + 1$} of a quasigroup $G$ is a finite sequence $(g_0, \dotsc, g_k)$ with a corresponding parenthesization $P$ such that
$$
G = \left\{P\left(g_0, g_1^{\epsilon_1}, \dotsc, g_k^{\epsilon_k}\right) \, \middle| \, \epsilon_i \in \{0, 1\} \text{ for each } i \right\}
$$
The \emph{rank} of a quasigroup is the length of a minimum generating sequence.
If the quasigroup is a group, the operation is associative, so the parenthesization is superfluous, and the rank is simply the size of a minimum generating set.
If $S$ is a subset of group elements, the \emph{subgroup generated by $S$}, denoted $\langle S \rangle$, is the set of all group elements that can be expressed as a product of any number of elements of $S$.

The \textsc{Quasigroup Rank} problem is defined as follows.
Given the multiplication table of a quasigroup and an integer $k$ in binary, decide whether the rank of the quasigroup is $k$ or less.
The \textsc{Group Rank} problem is defined similarly.

The \textsc{Quasigroup Isomorphism} problem is in $\bFOLL$ \autocite[Theorem~3.4]{ctw13}.
Implicit in the algorithm there is a $\bFOLL$ algorithm for \textsc{Quasigroup Rank}.

\begin{theorem}[{Implicit in \autocite[Theorem~3.4]{ctw13}}]
  \textsc{Quasigroup Rank} is in $\bFOLL$.
\end{theorem}
\begin{proof}
  By \autocite[Theorem~3.3]{ctw13}, each finite quasigroup with $n$ elements has a generating sequence of size $O(\log n)$ with a parenthesization of depth $O(\log \log n)$.
  The $\bFOLL$ algorithm works as follows on input quasigroup $G$ with $n$ elements (given as a multiplication table) and integer $k$, guaranteed to be in $O(\log n)$.
  \begin{enumerate}
  \item Nondeterministically choose a sequence of quasigroup elements $(g_0, \dotsc, g_k)$ and a parenthesization $P$ of depth $O(\log k)$.
  \item For each quasigroup element $a$, in parallel:
    \begin{enumerate}
    \item For each binary sequence $(\epsilon_1, \dotsc, \epsilon_k) \in \{0, 1\}^k$, in parallel:
      \begin{enumerate}
      \item Check if $P\left(g_0, g_1^{\epsilon_1}, \dotsc, g_k^{\epsilon_k}\right) = a$.
      \end{enumerate}
    \item Output $1$ if and only if some $(\epsilon_1, \dotsc, \epsilon_k)$ produces $a$.
    \end{enumerate}
  \item Output $1$ if and only if all elements of the quasigroup were produced.
  \end{enumerate}
  Since each quasigroup element can be represented using $O(\log n)$ bits, and since $k$ is in $O(\log n)$, this algorithm uses $O(\log^2 n)$ bits of nondeterminism to guess the generating sequence and parenthesization.
  %% TODO define FOLL and beta2 FOLL
  In step 2, there are $n$ group elements so we create $n$ parallel subcircuits.
  In step 2(a), since $k$ is in $O(\log n)$ there are a polynomial number of binary sequences of length $k$, so we create a polynomial number of parallel subcircuits.
  The innermost subcircuit that decides whether $P(g_0, g_1^{\epsilon_1}, \dotsc, g_k^{\epsilon_k}) = a$ requires a $O(\log \log n)$ depth circuit to compute the parenthesized product, plus a constant depth circuit to perform the equality comparison.
  In step 2(b), we use a single \textsc{or} gate (with unbounded fan-in).
  In step 3, we use a single \textsc{and} gate (with unbounded fan-in).
  Overall, the total size of this circuit is polynomial in $n$ and the depth is $O(\log \log n)$.
  The correctness of the circuit follows from the fact stated at the beginning of the proof, so we conclude that \textsc{Quasigroup Rank} is in $\bFOLL$.
\end{proof}

This theorem gives an immediate improvement over the previous best upper bound for \textsc{Group Rank}, which was $\L^2$ \cite{lsz77} (see \cite[Proposition~3]{at06} for a brief description of the algorithm).
$\L^2$ is the class of languages decidable by a deterministic Turing machine using $O(\log^2 n)$ space.

\begin{corollary}\label{cor:grouprank}
  \textsc{Group Rank} is in $\bFOLL \cap \bL$.
\end{corollary}

This is an improvement because
\begin{align*}
  (\bFOLL \cap \bL) & \subseteq \bL \subseteq \NL \subseteq \AC^1 \subseteq \bAC^1, \\ %\subseteq \bNC^2 \subseteq \L^2,
  (\bFOLL \cap \bL) & \subseteq \bFOLL \subseteq \bAC^1, %\subseteq \bNC^2 \subseteq \L^2.
\end{align*}
and
$$
\bAC^1 \subseteq \bNC^2 \subseteq \L^2.
$$
(In the last inclusion we use the fact that $\bNC^2 \subseteq \L^2$ \cite[Lemma~3.1]{wolf94}.)

This also immediately improves the result of \cite[Theorem~7]{at06}, which shows \textsc{Nilpotent Group Rank} is in \P, since
$$
(\bFOLL \cap \bL) \subseteq \NL \subseteq \P.
$$

However, the relationship between \FOLL{} and \L{} remains unknown (the best inclusion known is the uninteresting inclusion $\FOLL \subseteq \AC^1$), so the relationship between $\bFOLL$ and $\bL$ is unknown as well.

\begin{proof}[Proof of \autoref{cor:grouprank}]
  Since a group is a quasigroup, \textsc{Group Rank} is in $\bFOLL$ because \textsc{Quasigroup Rank} is.
  Thus, it suffices to show a $\bL$ algorithm for \textsc{Group Rank}.
  The algorithm proceeds as follows on input group $G$ of order $n$ given as a multiplication table and integer $k$ guaranteed to be in $O(\log n)$.
  \begin{enumerate}
  \item Nondeterministically choose $S$, a subset of $G$ of cardinality $k$.
  \item Accept if and only if $v \in \langle S \rangle$ for all $v \in G$.
  \end{enumerate}
  $G$ has rank $k$ or less if and only if it has a generating set $S$ of size $k$ or less if and only if each element of $G$ is in the subgroup generated by $S$.
  Hence this algorithm is correct as long as the algorithm for deciding subgroup membership is correct.
  This algorithm uses $O(\log^2 n)$ bits of nondeterminism, since each group element can be represented using $O(\log n)$ bits, and $k$ is in $O(\log n)$.
  In step 2, enumerating each $v$ requires only $O(\log n)$ bits of space, since the space can be reused after each iteration.
  The subgroup membership problem is in \SL{} \cite[Section~3]{bm89}, and $\SL = \L$ \cite{reingold08}, so the algorithm for deciding subgroup membership can be implemented using only $O(\log n)$ space as well.
  % TODO define \bL
  Therefore this is a correct algorithm for deciding \textsc{Group Rank} in $\bL$.
\end{proof}

Although, the precise relationship between \FOLL{} (and between $\FOLL^2$) and \L{} is unknown, \FOLL{} does not contain any class containing the \textsc{Parity} problem.
Since \textsc{Parity} is in \L, we know \FOLL{} does not contain \L.
Stated in a slightly more general way, neither \FOLL{} nor $\FOLL^2$ can be hard under $\AC^0$ many-one reductions for any complexity class that contains \textsc{Parity} \cite[Proposition~2.1]{bklm01}.
This is true even when the circuit is augmented with a polylogarithmic number of nondeterministic gates \cite[Section~4]{ctw13}.
This gives an immediate improvement to the upper bound of the \textsc{Quasigroup Rank} problem.

\begin{theorem}
  \textsc{Quasigroup Rank} is not hard under $\AC^0$ many-one reductions for any complexity class containing \textsc{Parity}.
\end{theorem}

Specifically, \textsc{Quasigroup Rank} is not hard for any of the classes in the inclusion chain
$$
\ACC^0 \subseteq \TC^0 \subseteq \NC^1 \subseteq \L \subseteq \NL \subseteq (\LOGCFL \cup \DET).
$$

\section{Future work}

Is \textsc{Quasigroup Rank} in $\bL$?

Is \textsc{Quasigroup Rank} reducible to \textsc{Quasigroup Isomorphism}?
Is the converse true?

Certainly \textsc{Quasigroup Rank} reduces to the problem of computing a minimum generating sequence, by simply computing the length of the computed minimum generating sequence.
Is the converse true?
It seems unlikely that knowing the rank of a quasigroup would yield any information about the contents of a minimum generating sequence.

\section{About this work}

Copyright 2014 Jef{}frey Finkelstein.

This work is licensed under the Creative Commons Attribution-ShareAlike 4.0 International License.
To view a copy of this license, visit the website \mbox{\url{https://creativecommons.org/licenses/by-sa/4.0/}}.

The \LaTeX{} markup which generated this document is available on its website \mbox{\url{https://github.com/jfinkels/grouprank}}.
The markup is distributed under the same license.

The author can be contacted via email at \email{jeffreyf@bu.edu}.

\printbibliography

\end{document}
