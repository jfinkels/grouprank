\begin{abstract}
% Foreword
%
% context (focus on anyone) why now? - current situation, and why the need is so important
The rank of a finite algebraic structure with a single binary operation is the minimum number of elements needed to express every other element under the closure of the operation.
% need (focus on readers) why you? - why this is relevant to the reader, and why something needed to be done
In the case of groups, the previous best algorithm for computing rank used polylogarithmic space.
%%% relevant existing work, given as part of the need
% task (focus on author) why me? - what was undertaken to address the need
We reduce the best upper bounds on the complexity of computing rank for groups and for quasigroups. %%, and provide a theoretically efficient algorithm for the latter.
% object (focus on document) why this document - what the document covers
This paper proves that the rank problem for these algebraic structures can be verified by highly restricted models of computation given only very short certificates of correctness.

% Summary
%
% findings (focus on author) what? - what the work revealed when performing the task
Specifically, we prove that the problem of deciding whether the rank of a finite quasigroup, given as a Cayley table, is smaller than a specified number is decidable by a circuit of depth $O(\log \log n)$ augmented with $O(\log^2 n)$ nondeterministic bits (the complexity class of problems decidable by such circuits is denoted $\bFOLL$).
Furthermore, if the quasigroup is a group, then the problem is also decidable by a Turing machine using $O(\log n)$ space and $O(\log^2 n)$ bits of nondeterminism with the ability to read the nondeterministic bits multiple times (the complexity class for problems like this is denoted $\bL$).
Finally, we provide similar results for related problems for other kinds of rank and relaxed notions of generating sets.
% conclusion (focus on readers) so what? - what the findings mean for the audience
These new upper bounds are significant improvements, especially for groups.
% perspective (focus on anyone) what now? - what should be done next
In general, the lens of limited nondeterminism provides an easy way to improve many simple algorithms, like the ones presented here, and we suspect it will be especially useful for other algebraic algorithms.
\end{abstract}
