\section{Computation of submagma membership}
% Foreword
%
% context (focus on anyone) why now? - current situation, and why the need is so important
As stated in the introduction, computing the rank of a magma reduces to the problem of deciding submagma membership.
% need (focus on readers) why you? - why this is relevant to the reader, and why something needed to be done
Thus, in order to determine an upper bound on the complexity of computing the rank, we can determine the complexity of deciding membership.
%
%%% relevant existing work, given as part of the need
Fortunately, the membership problem for a submagma when the magma is given as its Cayley table is a well-studied problem.
% task (focus on author) why me? - what was undertaken to address the need
% object (focus on document) why this document - what the document covers
This section reviews the complexity for the membership problem for magmas, semigroups, quasigroups, and groups.

% Summary
%
% findings (focus on author) what? - what the work revealed when performing the task
We recall that for magmas the membership problem, given the Cayley table, is in $\NP$ and for semigroups $\NL$.
We prove that for quasigroups the problem is in $\FOLL$ and for groups $\L$.
% conclusion (focus on readers) so what? - what the findings mean for the audience
These upper bounds allow us to prove upper bounds on the rank problem in the next section.
% perspective (focus on anyone) what now? - what should be done next
If future work reveals more efficient algorithms for the membership problem for quasigroups or groups, we can provide improved algorithms for computing the rank of these algebraic structures.

The \textsc{Submagma Membership} problem is defined as follows.
The inputs are a magma $G$ given as a Cayley table, a magma element $h$, and a finite set of magma elements $S$.
The problem is to decide whether $h \in \gen{S}$, that is, whether $h$ is in the submagma generated by $S$.

\begin{lemma}[{\autocite[Corollary~9]{jl76}}]\label{lem:submagmamem}
  \textsc{Submagma Membership} is $\P$-complete.
\end{lemma}
%% \begin{proof}
%%   Suppose the magma has $n$ elements and $h$ is the element to check for submagma membership.
%%   The certificate is a sequence of $n$ elements chosen from $S$ and a parenthesization $P$ of that sequence.
%%   (We choose a sequence of $n$ elements from $S$ because in the worst case, the magma is the cyclic group of order $n$, the set $S$ is the singleton set $\{1\}$, and $h$ is $1^n$.)
%%   The verification procedure computes the parenthesized product and checks that it is $h$.

%%   Each element can be represented with $O(\log n)$ bits, so a sequence of $n$ such elements can be represented with $O(n \log n)$ bits.
%%   The parenthesization can be represented as a binary string of length $n$.
%%   Thus the overall size of the certificate is $O(n \log n)$.
%% \end{proof}

The \textsc{Subsemigroup Membership} problem is defined as follows.
The inputs are a semigroup $G$ given as a Cayley table, a semigroup element $h$, and a finite set $S$ of semigroup elements.
The problem is to decide whether $h \in \gen{S}$, that is, whether $h$ is in the subsemigroup generated by $S$.

\begin{lemma}[{\autocite{jll76}}]\label{lem:subsemigroupmem}
  \textsc{Subsemigroup Membership} is $\NL$-complete.
\end{lemma}
%% \begin{proof}
%%   We will show a logarithmic space disjunctive truth-table reduction from \textsc{Subsemigroup Membership} to \textsc{Directed Path}.
%%   Since the latter problem is in $\NL$ and $\NL$ is closed under this type of reduction, we conclude that the former problem is in $\NL$ as well.

%%   Suppose $G$ is the semigroup, $h$ is the semigroup element, and $S$ is the set of subsemigroup generators described in the problem definition.
%%   The Cayley table induces an edge-labeled directed multigraph $\Gamma$ with self-loops: the vertices of the graph are the semigroup elements, and for each semigroup element $a$, $b$, and $c$, there is an edge labeled $b$ from $a$ to $c$ if $a \cdot b = c$.
%%   Consider the subgraph $\Gamma_S$ induced by those edges whose labels are in $S$.
%%   The element $h$ is in $\gen{S}$ if and only if there is a path from some vertex in $S$ to $h$.
%%   In other words, if $S = \{s_1, \dotsc, s_m\}$ for some natural number $m$,
%%   %% and $\Path(\Gamma_S, h, s_i)$ is a proposition that is true exactly when there is a directed path from $h$ to $s_i$ in $\Gamma_S$,
%%   \begin{equation*}
%%     h \in \gen{S} \iff \bigvee_{i = 1}^m (\Gamma_S, h, s_i) \in \textsc{Directed Path}.
%%   \end{equation*}
%%   Therefore we choose the reduction to be $(G, h, S) \mapsto ((\Gamma_S, h, s_1), \dotsc, (\Gamma_S, h, s_m))$.

%%   Computing the adjacency matrix of $\Gamma_S$ from the Cayley table of $G$ and the set $S$ can be done in logarithmic space by setting each entry $(x, y)$ in the Cayley table to be zero if $y$ is not in $S$.
%%   Iterating over all edges in the adjacency matrix requires logarithmic space, and enumerating each of the $m$ outputs requires space logarithmic in $m$, which is logarithmic in $n$ because $m \leq n$.
%%   We conclude that the reduction is a correct logarithmic space disjunctive truth-table reduction.
%% \end{proof}

The \textsc{Cube Membership} problem is defined as follows.
The inputs are a quasigroup $G$ given as a Cayley table, a quasigroup element $h$, a finite \emph{sequence} of quasigroup elements $S$, and a parenthesization $P$ for that sequence.
The problem is to decide whether $h \in \cube_P(S)$.

\begin{lemma}[{Implicit in \autocite[Theorem~3.4]{ctw13}}]\label{lem:cubemem}
  \textsc{Cube Membership} is decidable by an $\L$-uniform family of unbounded fan-in circuits with size $O(2^k k n^2 \log n)$ and depth $O(d)$, where $n$ is the order of the quasigroup, $k$ is the size of the generating sequence, and $d$ is the depth of the parenthesization.

  %In particular, if $k$ is in $O(\log n)$ and $d$ is in $O(\log \log n)$, then the circuit is of size polynomial in $n$ and of depth $O(\log \log n)$, that is, \textsc{Cube Membership} is in $\FOLL$.
  In particular, if $k = O(\log n)$ and $d = O(\log \log n)$, then \textsc{Cube Membership} is in $\FOLL$.
\end{lemma}
\begin{proof}
  The input to the circuit is the Cayley table for a quasigroup, a quasigroup element $h$, a generating sequence $S$, and a parenthesization $P$.
  Suppose $S = (g_0, \dotsc, g_k)$ for some positive integer $k$.
  Since the circuit needs to determine if $h$ is in $\cube_P(S)$, the circuit accepts if and only if there is some sequence of bits $(\epsilon_1, \dotsc, \epsilon_k)$ such that $h = P(g_0, g_1^{\epsilon_1}, \dotsc, g_k^{\epsilon_k})$.
  Thus the circuit consists of $2^k$ subcircuits joined to a single \textsc{or} gate, each subcircuit deciding whether one of the $2^k$ possible $k$-bit sequences $(\epsilon_1, \dotsc, \epsilon_k)$ produces $h$ under the given parenthesization.

  The subcircuit corresponding to binary sequence $(\epsilon_1, \dotsc, \epsilon_k)$ computes the parenthesized product $P(g_0, g_1^{\epsilon_1}, \dotsc, g_k^{\epsilon_k})$.
  Computing the parenthesized product can be implemented in $O(k n^2 \log n)$ size and $O(d)$ depth by Lemma~\ref{lem:product}.
  Comparing the element produced this way to the element $h$ can be done with a constant depth, $O(\log n)$ size equality comparison circuit.

  We conclude that the overall size of the circuit is $O(2^k k n^2 \log n)$ and the overall depth of the circuit is $O(d)$.
\end{proof}

Although the notion of cube generating sequence will give us a better upper bound for computing quasigroup rank, we prefer to consider the more natural notion of a generating set, as we did for magmas, semigroups, and quasigroups.
Since we know from Lemma~\ref{lem:small} that we need only consider candidate generating sets of size $O(\log n)$ and candidate parenthesizations of depth $O(\log \log n)$, we use a generalized form of the quasigroup membership problem that allows us to specify bounds on the generating set size and parenthesization depth.

The \textsc{Bounded Subquasigroup Membership} problem is defined as follows.
The inputs are a quasigroup $G$ given as a Cayley table, a quasigroup element $h$, a finite set $S$ of quasigroup elements, a positive integer $k$, and a positive integer $d$.
The problem is to decide whether there is a sequence $s$ in $S^k$ and a parenthesization of depth $d$ on $k$ elements such that $h = P(s)$.
(This is the definition of ``$h \in \gen{S}$'', but with specific size and depth bounds on the binary tree that generates $h$.)
This problem should be at least as difficult as \textsc{Cube Membership}: the former requires finding an appropriate sequence and parenthesization, whereas for the latter, they are fixed beforehand.

\begin{lemma}\label{lem:subquasigroupmem}
  \textsc{Bounded Subquasigroup Membership} is decidable by an $\L$-uniform family of unbounded fan-in circuits with size $O(n^2 k \log n)$ and depth $O(d)$, using $O(k \log n)$ nondeterministic bits.

  In particular, if $k = O(\log n)$ and $d = O(\log \log n)$, then \textsc{Bounded Subquasigroup Membership} is in $\bFOLL$.
\end{lemma}
\begin{proof}
  The algorithm is similar to that of Lemma~\ref{lem:cubemem}, except now we must nondeterministically choose a sequence and parenthesization.
  The circuit nondeterministically chooses $k$ elements of $S$ and a parenthesization of $k$ elements of depth $d$, then accepts if and only if that parenthesized product is $h$.
  Choosing $k$ elements, each of size $O(\log n)$, requires $O(k \log n)$ bits and choosing a parenthesization requires $O(k)$ bits, so the total number of nondeterministic bits required is $O(k \log n)$.
  By Lemma~\ref{lem:product}, computing the parenthesized product requires a circuit of size $O(k n^2 \log n)$ and depth $O(d)$.
  The final equality comparison requires size $O(\log n)$ and depth $O(1)$.
  We conclude that the overall size of the circuit is $O(k n^2 \log n)$ and the overall depth of the circuit is $O(d)$.
\end{proof}

The \textsc{Subgroup Membership} problem is defined as follows.
The inputs are a group $G$ given as a Cayley table, a group element $h$, and a finite set $S$ of group elements.
The problem is to decide whether $h \in \gen{S}$.

\begin{lemma}\label{lem:subgroupmem}
  \textsc{Subgroup Membership} is in $\L$.
\end{lemma}
\begin{proof}
  The problem is in $\SL$ by a reduction to \textsc{Undirected Path} \autocite[Section~3]{bm89}, and $\SL = \L$ \cite{reingold08}.
\end{proof}
