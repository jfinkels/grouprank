\section{Computation of magma rank}\label{sec:rank}

% Foreword
%
% context (focus on anyone) why now? - current situation, and why the need is so important
Computing submagma membership is where most of the work occurs.
% need (focus on readers) why you? - why this is relevant to the reader, and why something needed to be done
Now we need only reduce the rank problem to the membership problem.
% task (focus on author) why me? - what was undertaken to address the need
We do this via a truth-table reduction of relatively low complexity.
% object (focus on document) why this document - what the document covers
This section uses these reductions and the results of the previous section to prove the upper bounds on the rank problem as advertised in the introduction.

% Summary
%
% findings (focus on author) what? - what the work revealed when performing the task
Theorem~\ref{thm:rank} proves that for semigroups and magmas, the rank problem is in $\NP$ (which was already known), for quasigroups $\bFOLL$, and for groups $\bFOLL \cap \bL$.
% conclusion (focus on readers) so what? - what the findings mean for the audience
This means that for groups and quasigroups, the problem can be verified quickly in parallel given a very short witness.
% perspective (focus on anyone) what now? - what should be done next
We conjecture that for magmas and semigroups, the problems are hard for their respective complexity classes.

The \textsc{Magma Rank} problem is defined as follows.
Given the Cayley table of a magma and an integer $k$ in unary, decide whether the rank of the magma is $k$ or less.
The restrictions of this problem to quasigroups, semigroups, and groups, respectively, are defined similarly.
The integer $k$ is given in unary in order to facilitate the construction of uniform circuit families that decide the problem; since the size of the Cayley table is $n^2 \log n$ and $k$ is always at most $n$, encoding the integer in unary does not cause an exponential increase in the size of the input to the problems.

The reductions in the following lemma are implicit in \autocite[Theorem~3.4]{ctw13}.
That theorem demonstrates a $\bFOLL$ algorithm for deciding whether two quasigroups are isomorphic, and the first part of that algorithm determines whether a given sequence of quasigroup elements with a parenthesization is a cube generating sequence.

\begin{lemma}\label{lem:ranktomem}
  \mbox{}
  \begin{enumerate}
  \item $\textsc{Magma Rank} \leq^{\NAC^0}_{ctt} \textsc{Submagma Membership}$.
  \item $\textsc{Semigroup Rank} \leq^{\NAC^0}_{ctt} \textsc{Subsemigroup Membership}$.
  \item $\textsc{Quasigroup Rank} \leq^{\bAC^0}_{ctt} \textsc{Cube Membership}$.
  \item $\textsc{Quasigroup Rank} \leq^{\bAC^0}_{ctt} \textsc{Bounded Subquasigroup Membership}$.
  \item $\textsc{Group Rank} \leq^{\bAC^0}_{ctt} \textsc{Subgroup Membership}$.
  \end{enumerate}
\end{lemma}
\begin{proof}[Proof sketch]
  Each reduction is essentially the same.
  By the definition of rank, we have for each magma $G$,
  \begin{equation*}
    \rank(G) \leq k \iff \exists S \subseteq G \colon |S| \leq k \land \bigwedge_{i = 1}^n g_i \in \gen{S},
  \end{equation*}
  with slight modifications to deal with sequences instead of sets, required parenthesizations, and/or cube membership.
  Thus the witness is the binary representation of the set $S$ and the function $f$ computing the reduction outputs the $n$-tuple $((G, g_1, S), \dotsc, (G, g_n, S))$.
  For the reduction to \textsc{Cube Membership}, the witness also includes a parenthesization, and the reduction must output that as well.
  For the reduction to \textsc{Bounded Subquasigroup Membership}, the witness includes also the depth $d$ of parenthesizations, and the reduction must output $d$ as well as the number $k$.

  The correctness of the reductions follows immediately from the definitions of rank, but the real interest is in the complexity.
  Unlike for quasigroups and groups (Lemma~\ref{lem:small} and Lemma~\ref{lem:log}), we have no general upper bound on the minimum size of a generating set for magmas.
  Thus, for magmas, the best we can do is nondeterministically choose a set of $k$ generators and determine if that set generates the magma, where $k$ can be as large as the number of elements in the magma.
  Since each magma element can be represented by $O(\log n)$ bits, the number of nondeterministic bits used is $O(n \log n)$.
  For semigroups, we apply the exact same analysis.

  By Lemma~\ref{lem:small}, it suffices to consider inputs to \textsc{Quasigroup Rank} in which $k$ is in $O(\log n)$ and inputs to \textsc{Cube Membership} in which $P$ is of depth $O(\log \log n)$.
  Therefore, the witness for that reduction is a sequence $S$ of $k$ quasigroup elements and a parenthesization $P$ of depth $O(\log \log n)$.
  Now the number of nondeterministic bits used is $O(\log^2 n)$, since $S$ is a set of $O(\log n)$ strings, each of length $O(\log n)$.
  For the reduction to \textsc{Bounded Subquasigroup Membership}, we still need $O(\log^2 n)$ bits to guess the generating set and the depth $d$, but we can assume $k = O(\log n)$ and $d = O(\log \log n)$, by Lemma~\ref{lem:small}.
  For groups, we invoke Lemma~\ref{lem:log}, which states that any group of order $n$ has a generating set of size at most $\log n$.
  Also, we don't need to guess a parenthesization (although we still use $O(\log^2 n)$ nondeterministic bits to guess the generating set).
\end{proof}

%% \todo{Show the probability of success for a randomized reduction to the membership problem for each of these algebraic structures.}

\begin{theorem}\label{thm:rank}
  \mbox{}
  \begin{enumerate}
  \item \textsc{Magma Rank} is in $\NP$.
  \item \textsc{Semigroup Rank} is in $\NP$.
  \item \textsc{Quasigroup Rank} is in $\bFOLL$.
  \item \textsc{Group Rank} is in $\bFOLL \cap \bL$.
  \end{enumerate}
\end{theorem}
\begin{proof}
  Follows from Lemma~\ref{lem:ctt} and Lemma~\ref{lem:ranktomem}, along with
  \begin{enumerate}
  \item Lemma~\ref{lem:submagmamem} for magmas,
  \item Lemma~\ref{lem:subsemigroupmem} for semigroups,
  \item Lemma~\ref{lem:cubemem} for quasigroups,
  \item Lemma~\ref{lem:subgroupmem} for groups.
  \end{enumerate}
  For groups the problem is also in $\bFOLL$ since a group is a quasigroup.
\end{proof}

%% These reductions can be generalized to the problem of computing the size of a minimum generating set for an arbitrary subset of the magma elements.
%% The \textsc{Generalized Magma Rank} problem is defined as follows (and there are analagous problems for the other algebraic structures).
%% Given a magma $G$ as a Cayley table, a finite set $T$ of magma elements, and a natural number $k$ in unary, decide whether there is a set $S$ of size at most $k$ such that $S \subseteq T \subseteq \gen{S}$.
%% \textsc{Magma Rank} occurs as a special case when choosing $T = G$.
%% Still, this problem reduces to the appropriate membership problem by the same reduction as in \autoref{lem:ranktomem}: nondeterministically choose a subset $S$ of $T$ with $|S| \leq k$, then decide whether each element of $T$ is in $\gen{S}$.

%% We can reprove \autocite[Theorem~3.4]{ctw13} using this strategy as well.
%% The alternate proof is a reduction from \textsc{Quasigroup Isomorphism} to the join of \emph{two} languages, \textsc{Cube Membership} and \textsc{Product Equality}.
%% The latter is the problem of deciding whether two parenthesized products are equal according to the quasigroup operation given as a Cayley table.
%% If the parenthesization is of depth $O(\log \log n)$, this problem is in $\FOLL$ by \autoref{lem:product}.

%% \begin{theorem}[{\autocite[Theorem~3.4]{ctw13}}]
%%   \textsc{Quasigroup Isomorphism} is in $\bFOLL$.
%% \end{theorem}
%% \begin{proof}
%%   This is a brief overview of the alternate proof.
%%   We will show a $\bAC^0$ conjunctive normal form truth-table reduction from $\textsc{Quasigroup Isomorphism}$ to the join of $\textsc{Cube Membership}$ and $\textsc{Product Equality}$, both of which are in $\FOLL$.
%%   Assuming this reduction exists, we conclude using a proof similar to that of \autoref{lem:ctt} that \textsc{Quasigroup Isomorphism} is in $\bFOLL$.

%%   The reduction first guesses two cube generating sequences, $\mathbf{g}$ for $G$ and $\mathbf{h}$ for $H$, both of length $O(\log n)$ and a parenthesization $P$ of depth $O(\log \log n)$, then outputs the conjunction of the following queries.
%%   \begin{align}
%%     & \bigwedge_{g \in G} g \in \cube_P(\mathbf{g}) \\
%%     & \bigwedge_{h \in H} h \in \cube_P(\mathbf{h}) \\
%%     & \; \bigwedge_{\mathclap{\epsilon, \eta, \nu \in \{0, 1\}^k}} \left(P(\mathbf{g}^\epsilon) = P(\mathbf{g}^\eta) \cdot P(\mathbf{g}^\nu) \iff P(\mathbf{h}^\epsilon) = P(\mathbf{h}^\eta) \cdot P(\mathbf{h}^\nu)\right)
%%   \end{align}
%%   The first two formulas ensure that $\mathbf{g}$ and $\mathbf{h}$ are cube generating sequences for $G$ and $H$, respectively.
%%   In the third formula, if $\mathbf{g} = (g_0, g_1, \dotsc, g_k)$, then $\mathbf{g}^\epsilon$ denotes $(g_0, g_1^{\epsilon_1}, \dotsc, g_k^{\epsilon_k})$.
%%   This formula checks that the bijection $g_i \mapsto h_i$ is a homomorphism.

%%   Each of the first two formulas comprises $n$ conjunctive queries to \textsc{Cube Membership}.
%%   The last formula comprises a polynomial number of queries in conjunctive normal form to \textsc{Product Equality}.
%%   Thus we have the required reduction.
%% \end{proof}

%%We conclude this section with a few observations about \autoref{thm:rank}.
%%First,
In this proof we did not use the reduction from \textsc{Quasigroup Rank} to \textsc{Bounded Subquasigroup Membership}, because the closure of $\bFOLL$ under $\bAC^0$ conjunctive truth-table reductions is $\NFOLL$, that is, $\FOLL$ with a polynomial amount of nondeterminism, whereas the closure of $\FOLL$ under the same reductions is $\bFOLL$, a subset of $\NFOLL$.

%% Second, a slight generalization of \autocite[Theorem~7]{py96} already proves that \textsc{Magma Rank} is in (and complete for) the class of problems decidable by a polynomial-time Turing machine with $O(n \log n)$ nondeterministic bits.
%% We have nevertheless included the fact that \textsc{Magma Rank} is in $\NP$ to highlight the general strategy for proving these upper bounds for each class of algebraic structure.

%% Third, we can almost show a reduction in the opposite direction of \autoref{lem:ranktomem}.
%% The \textsc{Submagma Rank} problem (a search problem) is defined as follows.
%% Given the Cayley table of a magma and a set of magma elements $S$, output the rank of $\gen{S}$.
%% (The \textsc{Submagma Rank} problem is more general than the \textsc{Magma Rank} problem: the latter reduces to the former by choosing $S = G$.)

%% \begin{proposition}
%%   \textsc{Submagma Membership} reduces to the \textsc{Submagma Rank} function by a nonadaptive $\AC^0$ Turing reduction making exactly two queries.
%% \end{proposition}
%% \begin{proof}
%%   We know that for any magma $G$, any magma element $h$, and any subset of magma elements $S$,
%%   \begin{equation*}
%%     h \in \gen{S} \iff \rank(\gen{S}) = \rank(\gen{S \cup \{h\}}).
%%   \end{equation*}
%%   (This is analogous to the corresponding situation in linear algebra: a vector $h$ is in the span of a set of vectors $S$ exactly when the rank of $S$ does not increase when $h$ is added to $S$.)
%%   Thus the generator of the reduction is the function $(G, h, S) \mapsto ((G, S), (G, S \cup \{h\}))$ and the evaluator of the reduction compares $\rank({S})$ to $\rank(\gen{S \cup \{h\}})$ for equality.
%% \end{proof}

%% However, this reduction is not satisfying, because the \textsc{Submagma Rank} is essentially the \textsc{Magma Rank} problem when the input is provided as a set of generators instead of as a Cayley table.
%% As stated in the reduction, this representation may be exponentially smaller than the Cayley table representation.

%% Fourth,
Although the precise relationship between $\FOLL$ and $\L$ is unknown, $\FOLL$ does not contain any class containing the \textsc{Parity} problem.
Since \textsc{Parity} is in $\L$, we know $\FOLL$ does not contain $\L$.
Stated in a slightly more general way, $\FOLL$ cannot be hard under $\AC^0$ many-one reductions for any complexity class that contains \textsc{Parity} \cite[Proposition~2.1]{bklm01}.
This is true even when the circuit is augmented with a polylogarithmic number of nondeterministic bits \cite[Section~4]{ctw13}.
This gives an immediate improvement to the upper bound of the \textsc{Quasigroup Rank} problem.

\begin{theorem}
  \textsc{Quasigroup Rank} is not hard under $\AC^0$ many-one reductions for any complexity class containing \textsc{Parity}.
\end{theorem}

Specifically, \textsc{Quasigroup Rank} is not hard for any of the classes in the inclusion chain
\begin{equation*}
  \ACC^0 \subseteq \TC^0 \subseteq \NC^1 \subseteq \L \subseteq \NL \subseteq (\LOGCFL \cup \DET).
\end{equation*}

%% Finally, we consider whether there is a randomized logarithmic space algorithm for \textsc{Group Rank}, which would immediately improve the upper bound of \autoref{thm:rank}.
%% Let $\RL$ be the class of languages $L$ for which there is a deterministic Turing machine with a two-way read-only tape for the input string, a one-way read-only tape for the random bits, and a two-way read-write work tape in which only $O(\log n)$ cells are used, such that $x \in L$ implies that at least half the binary strings $r$ of polynomial of length cause the machine to accept on input $x$ and random bits $r$ and $x \notin L$ implies that none of the binary strings $r$ cause the machine to accept.
%% Let $\rL$ be the class of languages for which the tape containing the random bits is \emph{two-way} read-only tape, and for which the length of the binary string $r$ is $O(\log^2 n)$.
%% By \autocite[Corollary~1]{nisan92} (see also \autocite{nisan90}), $\RL \subseteq \rL$, and by the definitions, $\rL \subseteq \bL$.
%% Now our question is whether \textsc{Group Rank} is in $\rL$, or even in $\RL$.
