\section{Computing other types of ranks}
\label{sec:otherranks}

% % Foreword %
%
% %% Context (anyone - why now?) %%
%
% What is the current situation, and why is the need so important?
%
The ``rank'' of a vector space is the number of vectors in any basis of the space.
In this setting, the vectors must not only span (or ``generate'') the space, but also be \emph{linearly independent}.
The rank as defined and used in the previous sections have no such independence requirement.
% %% Need (readers - why you?) %%
%
% Why is this relevant to the reader, and why does something need to be done?
% (Also reference relevant existing work.)
%
However, we could use other definitions of rank for groups, semigroups, etc., that do require independence in some way.
\autocite{hr99, hr00} define five such ranks for semigroups.
% %% Task (author - why me?) %%
%
% What was undertaken to address the need?
%
We apply the same analysis as in \autoref{sec:rank} with the new definition to get some similar results;
% %% Object (document - why this document?) %%
%
% What does this document cover?
%
this section summarizes those similarities and differences.

% % Summary %
%
% %% Findings (author - what?)
%
% What did the work reveal when performing the task?
%
\autoref{thm:otherranks} shows that for groups, most other rank definitions yield limited nondeterminism (or conondeterminism) algorithms with simple verifiers, similar to that of \autoref{thm:rank}.
% %% Conclusion (readers - so what?)
%
% What did the findings mean for the audience?
%
The limited nondeterminism lens is capable of improving algorithms for computational problems whose conditions seem even more strict than that of the standard rank problem.
% %% Perspective (anyone - what now?)
%
% What should be done next?
One definition of rank, the large rank, seems unlikely to be solvable with limited nondeterminism, and we conjecture this problem is $\NP$-complete.

Until now we have considered the most common definition of rank, the minimum cardinality of a generating set, but there are others.
Here are the five common definitions of rank, as described in \autocite{hr99, hr00}.
For any magma $G$,
\begin{align*}
  \rank_L(G) & = \min \{ k \, | \, \text{each subset of cardinality } k \text{ is a generating set}\} \\
  \rank_u(G) & = \max \{ |S| \, | \, S \subseteq G \text{ and } S \text{ is an independent set} \} \\
  \rank_i(G) & = \max \{ |S| \, | \, S \subseteq G \text{ and } S \text{ is an independent generating set} \} \\
  \rank_\ell(G) & = \min \{ |S| \, | \, S \subseteq G \text{ and } S \text{ is a generating set} \} \\
  \rank_s(G) & = \max \{ k \, | \, \text{each subset of cardinality } k \text{ is an independent set} \}
\end{align*}

These are called \emph{large rank}, \emph{upper rank}, \emph{intermediate rank}, \emph{lower rank}, and \emph{small rank}, respectively.
The lower rank is the notion of rank discussed in the previous sections of the paper.
For each finite magma $G$,
%% \begin{proposition}
%%   For each finite magma $G$,
$$
    \rank_s(G) \leq \rank_\ell(G) \leq \rank_i(G) \leq \rank_u(G) \leq \rank_L(G).
$$
%% \end{proposition}
%% \begin{proof}
%%   \autocite{hr00} states this chain of inequalities for semigroups; we show that it holds for finite magmas as well.

%%   The inequalities $\rank_\ell(G) \leq \rank_i(G) \leq \rank_u(G)$ follow from the fact that the collection of independent generating sets is the intersection of the collection of independent sets and the collection of generating sets.

%%   Next we prove that $\rank_s(G) \leq \rank_\ell(G)$.
%%   If $\rank_\ell(G) = |G|$, then we are done since $\rank_s(G)$ must be bounded above by $|G|$ by definition.%
%%   \footnote{A semigroup $G$ with $\rank_\ell(G) = |G|$ is sometimes called a ``royal semigroup'' (because it has the highest possible rank) \autocite{gh85}.}
%%   Thus it suffices to consider only magmas with $\rank_\ell(G) < |G|$.
%%   Assume with the intention of producing a contradiction that $\rank_s(G) > \rank_\ell(G)$.
%%   Let $S$ be the generating set of cardinality $\rank_\ell(G)$.
%%   Let $x \in G \setminus S$, which exists because $\rank_\ell(G) < |G|$.
%%   Now $S \cup \{x\}$ is a generating set of cardinality at most $\rank_s(G)$.
%%   By definition of $\rank_s(G)$, this means $S \cup \{x\}$ is independent.
%%   But $x \in \gen{S}$ since $S$ generates $G$, so $S$ cannot be independent.
%%   This is a contradiction, therefore $\rank_s(G) \leq \rank_\ell(G)$.

%%   Finally we prove that $\rank_u(G) \leq \rank_L(G)$.
%%   If $\rank_L(G) = |G|$, then $G$ is the smallest generating set, so $\rank_\ell(G) = \rank_L(G)$, which subsumes the equality $\rank_u(G) = \rank_L(G)$.
%%   Thus it suffices to consider only magmas with $\rank_L(G) < |G|$.
%%   Assume with the intention of producing a contradiction that $\rank_u(G) > \rank_L(G)$.
%%   Let $S$ be the independent set of cardinality $\rank_u(G)$ and let $x$ be an arbitrary element of $S$.
%%   Then the set $S \setminus \{x\}$ has cardinality at least $\rank_L(G)$, so it is a generating set for $G$.
%%   This means $x \in \gen{S \setminus \{x\}}$, a contradiction with the hypothesis that $S$ is independent.
%%   Therefore, $\rank_u(G) \leq \rank_L(G)$.
%% \end{proof}

The framework of \autoref{sec:rank} provides a simple way of determining the complexity of computing these functions.
The decision problems corresponding to these rank functions are defined similar to the rank problems above.
%% \begin{align*}
%%   \textsc{Magma Large Rank} & = \{ (G, k) \, | \, \rank_L(G) \leq k \} \\
%%   \textsc{Magma Upper Rank} & = \{ (G, k) \, | \, \rank_u(G) \geq k \} \\
%%   \textsc{Magma Intermediate Rank} & = \{ (G, k) \, | \, \rank_i(G) \geq k \} \\
%%   \textsc{Magma Lower Rank} & = \{ (G, k) \, | \, \rank_\ell(G) \leq k \} \\
%%   \textsc{Magma Small Rank} & = \{ (G, k) \, | \, \rank_s(G) \geq k \}
%% \end{align*}
Some are maximization problems and some are minimization problems, depending on whether the rank is a minimum or a maximum.
%% The problems are defined similarly for the other algebraic structures.

We show how to construct a nondeterministic reduction from the upper rank problem, but the technique can be applied to each other type of rank problem by replacing existence with universal quantification where necessary and by using the appropriate logical formula for ``is generating'' or ``is independent''.
Let $G^k$ denote the collection of subsets of $G$ of cardinality $k$.
%% \begin{align*}
%%   \rank_L(G) \leq k & \iff \forall S \subseteq G^k\colon S \text{ is a generating set} \\
%%   \rank_u(G) \geq k & \iff \exists S \subseteq G^k\colon S \text{ is independent} \\
%%   \rank_i(G) \geq k & \iff \exists S \subseteq G^k\colon S \text{ is an independent generating set} \\
%%   \rank_\ell(G) \leq k & \iff \exists S \subseteq G^k\colon S \text{ is a generating set} \\
%%   \rank_s(G) \geq k & \iff \forall S \subseteq G^k\colon S \text{ is an independent set}
%% \end{align*}
Since $S$ is an independent set exactly when $x \notin \gen{S \setminus \{x\}}$ for each $x$ and $S$ is a generating set for $G$ exactly when $g \in \gen{S}$ for each $g$ in $G$, these reductions are nondeterministic or conondeterministic truth-table reductions.
$$
  \rank_u(G) \geq k \iff \exists S \subseteq G^k\colon \bigwedge_{x \in S} x \notin \gen{S \setminus \{x\}}
$$
%% \begin{align*}
%%   \rank_L(G) \leq k & \iff \forall S \subseteq G^k\colon \bigwedge_{g \in G} g \in \gen{S} \\
%%   \rank_u(G) \geq k & \iff \exists S \subseteq G^k\colon \bigwedge_{x \in S} x \notin \gen{S \setminus \{x\}} \\
%%   \rank_i(G) \geq k & \iff \exists S \subseteq G^k\colon \bigwedge_{g \in G} g \in \gen{S} \land \bigwedge_{x \in S} x \notin \gen{S \setminus \{x\}} \\
%%   \rank_\ell(G) \leq k & \iff \exists S \subseteq G^k\colon \bigwedge_{g \in G} g \in \gen{S} \\
%%   \rank_s(G) \geq k & \iff \forall S \subseteq G^k\colon \bigwedge_{x \in S} x \notin \gen{S \setminus \{x\}}
%% \end{align*}

By \autoref{lem:magind}, the upper rank of any finite group is $\log n$, so the small, lower, and intermediate ranks have an upper bound of $\log n$ as well.
(For large rank, however, it seems that any bound must depend on the prime factorization of $n$.)

For any language $L$, let $\overline{L}$ denote the complement of $L$ in $\Sigma^*$.
For any bit $b$, let $Lb$ denote the set $\{ wb \, | \, w \in L, b \in \Sigma\}$.
For any languages $L_0$ and $L_1$, let $L_0 \oplus L_1$ denote the \emph{join} of $L_0$ and $L_1$, defined by $L_0 \oplus L_1 = L_0 0 \cup L_1 1$.

\begin{lemma}
  For brevity, let \textsc{SM} denote \textsc{Subgroup Membership}.
  \mbox{}
  \begin{enumerate}
  \item $\textsc{Group Large Rank} \leq_{ctt}^{\co\NAC^0} \textsc{SM}$.
  \item $\textsc{Group Upper Rank} \leq_{ctt}^{\bAC^0} \overline{\textsc{SM}}$.
  \item $\textsc{Group Intermediate Rank} \leq_{ctt}^{\bAC^0} \textsc{SM}$.
  \item $\textsc{Group Lower Rank} \leq_{ctt}^{\bAC^0} \textsc{SM} \oplus \overline{\textsc{SM}}$.
  \item $\textsc{Group Small Rank} \leq_{ctt}^{\co\bAC^0} \overline{\textsc{SM}}$.
  \end{enumerate}
\end{lemma}

Similar reductions can be shown for the other algebraic structures.

The fact that $\L$ is closed under complement yields the following upper bound for computing the various types of rank.

\begin{theorem}\label{thm:otherranks}
  \mbox{}
  \begin{enumerate}
  \item $\textsc{Group Large Rank}$ is in $\coNP$.
  \item \textsc{Group Upper Rank}, \textsc{Group Intermediate Rank}, and \textsc{Group Lower Rank} are each in $\bL$.
  \item $\textsc{Group Small Rank}$ is in $\co\bL$.
  \end{enumerate}
\end{theorem}

If the Immerman–Szelepcsényi theorem \autocite{immerman88, szelepcsenyi88}, which proves $\NL = \coNL$, can be adapted to show $\bL = \co\bL$, then small rank problem can be decided in $\bL$ as well.
