\section{Computing other types of ranks}

\todo{Introduction to this section.}

Until now we have considered the most common definition of rank, the minimum cardinality of a generating set, but there are others.
Here are the five common definitions of rank, as described in \autocite{hr99, hr00}.
For any magma $G$,
\begin{align*}
  \rank_L(G) & = \min \{ k \, | \, \text{each subset of cardinality } k \text{ is a generating set}\} \\
  \rank_u(G) & = \max \{ |S| \, | \, S \subseteq G \text{ and } S \text{ is an independent set} \} \\
  \rank_i(G) & = \max \{ |S| \, | \, S \subseteq G \text{ and } S \text{ is an independent generating set} \} \\
  \rank_\ell(G) & = \min \{ |S| \, | \, S \subseteq G \text{ and } S \text{ is a generating set} \} \\
  \rank_s(G) & = \max \{ k \, | \, \text{each subset of cardinality } k \text{ is an independent set} \}
\end{align*}

These are called \emph{large rank}, \emph{upper rank}, \emph{intermediate rank}, \emph{lower rank}, and \emph{small rank}, respectively.
The lower rank is the notion of rank discussed in the previous sections of the paper.
\begin{proposition}
  For any finite semigroup $G$ (\todo{how about quasigroups or magmas?}),
  \begin{equation*}
    \rank_s(G) \leq \rank_\ell(G) \leq \rank_i(G) \leq \rank_u(G) \leq \rank_L(G).
  \end{equation*}
\end{proposition}

The framework of \autoref{sec:rank} provides a simple way of determining the complexity of computing these functions.
The decision problems corresponding to these rank functions are defined as follows.
\begin{align*}
  \textsc{Magma Large Rank} & = \{ (G, k) \, | \, \rank_L(G) \leq k \} \\
  \textsc{Magma Upper Rank} & = \{ (G, k) \, | \, \rank_u(G) \geq k \} \\
  \textsc{Magma Intermediate Rank} & = \{ (G, k) \, | \, \rank_i(G) \geq k \} \\
  \textsc{Magma Lower Rank} & = \{ (G, k) \, | \, \rank_\ell(G) \leq k \} \\
  \textsc{Magma Small Rank} & = \{ (G, k) \, | \, \rank_s(G) \geq k \}
\end{align*}
Some of these are maximization problems and some are minimization problems, depending on whether the rank is a minimum or a maximum.
The problems are defined similarly for the other algebraic structures.

We can construct nondeterministic or conondeterministic reductions as follows.
Let $G^k$ denote the collection of subsets of $G$ of cardinality $k$.
\begin{align*}
  \rank_L(G) \leq k & \iff \forall S \subseteq G^k\colon S \text{ is a generating set} \\
  \rank_u(G) \geq k & \iff \exists S \subseteq G^k\colon S \text{ is independent} \\
  \rank_i(G) \geq k & \iff \exists S \subseteq G^k\colon S \text{ is an independent generating set} \\
  \rank_\ell(G) \leq k & \iff \exists S \subseteq G^k\colon S \text{ is a generating set} \\
  \rank_s(G) \geq k & \iff \forall S \subseteq G^k\colon S \text{ is an independent set}
\end{align*}
Since $S$ is an independent set exactly when $x \notin \gen{S \setminus \{x\}}$ for each $x$ and $S$ is a generating set for $G$ exactly when $g \in \gen{S}$ for each $g$ in $G$, these reductions are nondeterministic or conondeterministic truth-table reductions.
\begin{align*}
  \rank_L(G) \leq k & \iff \forall S \subseteq G^k\colon \bigwedge_{g \in G} g \in \gen{S} \\
  \rank_u(G) \geq k & \iff \exists S \subseteq G^k\colon \bigwedge_{x \in S} x \notin \gen{S \setminus \{x\}} \\
  \rank_i(G) \geq k & \iff \exists S \subseteq G^k\colon \bigwedge_{g \in G} g \in \gen{S} \land \bigwedge_{x \in S} x \notin \gen{S \setminus \{x\}} \\
  \rank_\ell(G) \leq k & \iff \exists S \subseteq G^k\colon \bigwedge_{g \in G} g \in \gen{S} \\
  \rank_s(G) \geq k & \iff \forall S \subseteq G^k\colon \bigwedge_{x \in S} x \notin \gen{S \setminus \{x\}}
\end{align*}

By \autoref{lem:magind}, the upper rank of any finite group is $\log n$, so the small, lower, and intermediate ranks have an upper bound of $\log n$ as well.
(For large rank, however, it seems that any bound must depend on the prime factorization of $n$.)

For any language $L$, let $\overline{L}$ denote the complement of $L$ in $\Sigma^*$.
For any bit $b$, let $Lb$ denote the set $\{ wb \, | \, w \in L, b \in \Sigma\}$.
For any languages $L_0$ and $L_1$, let $L_0 \oplus L_1$ denote the \emph{join} of $L_0$ and $L_1$, defined by $L_0 \oplus L_1 = L_0 0 \cup L_1 1$.

\begin{lemma}
  For brevity, let \textsc{SM} denote \textsc{Subgroup Membership}.
  \mbox{}
  \begin{enumerate}
  \item $\textsc{Group Large Rank} \leq_{ctt}^{\co\NAC^0} \textsc{SM}$.
  \item $\textsc{Group Upper Rank} \leq_{ctt}^{\bAC^0} \overline{\textsc{SM}}$.
  \item $\textsc{Group Intermediate Rank} \leq_{ctt}^{\bAC^0} \textsc{SM}$.
  \item $\textsc{Group Lower Rank} \leq_{ctt}^{\bAC^0} \textsc{SM} \oplus \overline{\textsc{SM}}$.
  \item $\textsc{Group Small Rank} \leq_{ctt}^{\co\bAC^0} \overline{\textsc{SM}}$.
  \end{enumerate}
\end{lemma}

Similar reductions can be shown for the other algebraic structures.

The fact that $\L$ is closed under complement yields the following upper bound for computing the various types of rank.

\begin{theorem}
  \mbox{}
  \begin{enumerate}
  \item $\textsc{Group Large Rank}$ is in $\coNP$.
  \item $\textsc{Group Upper Rank}$ is in $\bL$.
  \item $\textsc{Group Intermediate Rank}$ is in $\bL$.
  \item $\textsc{Group Lower Rank}$ is in $\bL$.
  \item $\textsc{Group Small Rank}$ is in $\co\bL$.
  \end{enumerate}
\end{theorem}

If the Immerman–Szelepcsényi theorem \autocite{immerman88, szelepcsenyi88}, which proves $\NL = \coNL$, can be adapted to show $\bL = \co\bL$, then small rank problem can be decided in $\bL$ as well.
